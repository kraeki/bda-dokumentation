\chapter{Testing}

Dieses Dokument beschreibt wie sichergestellt wird, dass das System sich so verhaltet, wie es spezifiziert wird.

\subsection{Teststrategie}
Das System hat einen hoch komplexen Aufgbau und viele einzelne Komponenten. Umso wichtiger und komplizierter ist auch das Testing. Das Projekt wird auf drei Ebenen getestet. 

\vspace{.3cm}
\begin{description}
    \item[Unittests] Diese Tests sind vorallem während der Entwicklung einer Komponente wichtig.
    \item[Integrationstests] Eingige Komponenten sind stark von der Blockchain abhängig und müssen daher mit einer Blockchain-Instance getestet werden. Um dies zu vereinfachen wurde ein Dockerimage für eine lokale Blockchain erstellt  
    \item [Systemtests] Diese Tests testen das ganze System, inklusive aller Komponenten und der Blockchain. Mittels diesen Tests werden alle Use Cases validiert. Diese Tests gelten auch als Abnahmetests.
\end{description}

In den nachfolgenden Kapiteln werden die einzlnen Unittests, Integrationstests und Systemtests aufgelistet.

\subsection{Unittest}

\subsubsection{Smart Contract}
Die Tests können mit dem Befehl \emph{truffle test} auseführt werden.

\begin{table}[H]
\centering
\caption{Unittests des Smart Contracts Rentable}
\label{my-label}
\begin{tabular}{@{}L{.5cm}L{6cm}L{6cm}@{}}
\toprule
Nr. & 
Testname & 
Beschreibung 
\\ \midrule
1
& costPerSecond contract should be 7 
& Testet das Setzen von costPerSecond und dessen Berechnung 
\\ \midrule
2   
& descritption of contract should be 'leDescription'         
& Testet das \emph{Description}-Feld
\\ \midrule
3   
& location of contract should be 'leLocation'         
& Testet das \emph{Location}-Feld 
\\ \midrule
4   
& deposit of contract should be 500"         
& Testet das \emph{deposit}-Feld 
\\ \midrule
5   
& reserve (rent) the rentable         
& Testet die die \emph{reservedBetween} Funktion 
\\ \midrule
6   
& rent and check if reserved between         
& Testet die die \emph{rent} Funktion 
\\ \bottomrule
\end{tabular}
\end{table}

\begin{table}[H]
\centering
\caption{Unittests des Smart Contracts RentableDiscovery}
\label{my-label}
\begin{tabular}{@{}L{.5cm}L{6cm}L{6cm}@{}}
\toprule
Nr. & 
Testname & 
Beschreibung 
\\ \midrule
1
& can create contract using factory method
& Testet das Erzeugen von neuen \emph{Rentables} mittels Discovery Contract
\\ \bottomrule
\end{tabular}
\end{table}
\subsection{Integrationstests}
Die Integrationstests testen die einzelnen Komponenten gegen die Blockchain. 

\subsubsection{Doorman}

\paragraph{Voraussetzungen}

Um die Integrationstests von Doorman auszuführen muss folgendes Setup durchgeführt werden:

\begin{itemize}
    \item Auschecken der Doorman Resourcen
    \item Installieren der Abhängigkeiten
    \item Starten des Ethereum Nodes mittels Docker
    \item Veröffentlichen des Smart Contracts via truffle 
\end{itemize}

Danach können die Testfälle durchgespielt werden.
    
\begin{table}[H]
\centering
\caption{Test \#1: Starten von Doorman}
\label{my-label}
\begin{tabular}{@{}L{1.6cm}L{11cm}@{}}
\toprule
\textbf{Test \#1}
& Starten von Doorman mit Referenz zum veröffentlichten Contract
\\ \midrule
\textbf{Ablauf}
& 
\begin{enumerate}
    \item Eintragen der Addresse des veröffentlichten Contracts im Config file von Doorman
    \item Starten des Doormans
\end{enumerate}
\\ \midrule
\textbf{Ergebnis}
& Doorman startet und gibt folgendes Resultat auf der Konsole aus:
\begin{verbatim}
    Hoi ich bis der Dominik
\end{verbatim}
\\ \bottomrule
\end{tabular}
\end{table}

\begin{table}[H]
\centering
\caption{Test \#2: Doorman emfpängt Lock/Unlock-Befehl vom aktuellen Mieter}
\label{my-label}
\begin{tabular}{@{}L{1.6cm}L{11cm}@{}}
\toprule
\textbf{Test \#2}
& Doorman emfpängt Lock/Unlock-Befehl vom aktuellen Mieter
\\ \midrule
\textbf{Ablauf}
& 
\begin{enumerate}
    \item Mittels Truffle das Rentable für 5 min mieten
    \item Danach den Lock-Befehl ausführen
    \item Danach den Unlock-Befehl ausführen
\end{enumerate}
\\ \midrule
\textbf{Ergebnis}
& Doorman startet und gibt folgedes Resultat auf der Konsole aus:
\begin{verbatim}
    Hoi ich bis der Dominik
\end{verbatim}
\\ \bottomrule
\end{tabular}
\end{table}

\begin{table}[H]
\centering
\caption{Test \#3: Doorman emfpängt Lock/Unlock-Befehl von unbekanntem Account}
\label{my-label}
\begin{tabular}{@{}L{1.6cm}L{11cm}@{}}
\toprule
\textbf{Test \#3}
& Doorman emfpängt Lock/Unlock-Befehl von unbekanntem Account
\\ \midrule
\textbf{Ablauf}
& 
\begin{enumerate}
    \item Neuen Account auf der Node erstellen
    \item Lock Befehl signiert mit dem neuen Account absenden
    \item Unlock Befehl signiert mit dem neuen Account absenden
\end{enumerate}
\\ \midrule
\textbf{Ergebnis}
& Doorman startet und gibt folgedes Resultat auf der Konsole aus:
\begin{verbatim}
    Hoi ich bis der Dominik
\end{verbatim}
\\ \bottomrule
\end{tabular}
\end{table}

\subsubsection{Webapp}

\paragraph{Voraussetzungen}

Um die Integrationstests von der Webapp auszuführen muss folgendes Setup durchgeführt werden:

\begin{itemize}
    \item Auschecken der Webapp Resourcen
    \item Installieren der Abhängigkeiten
    \item Starten des Ethereum Nodes mittels Docker
    \item Starten der Webapp
    \item Veröffentlichen des Smart Contracts via truffle 
\end{itemize}

Danach können die Testfälle durchgespielt werden.


\begin{table}[H]
\centering
\caption{Test \#1: Hinzufügen eines Rentables }
\label{my-label}
\begin{tabular}{@{}L{1.6cm}L{11cm}@{}}
\toprule
\textbf{Test \#1}
& Hinzufügen eines Rentables
\\ \midrule
\textbf{Ablauf}
& 
\begin{enumerate}
    \item Füge das Rentable hinzu
    \item Schliesse den Browser und starte erneut
    \item Überprüfe, ob das Rentable noch hinzugefügt ist
\end{enumerate}
\\ \midrule
\textbf{Ergebnis}
& Das Rentable ist immer noch hinzugefügt.
\\ \bottomrule
\end{tabular}
\end{table}


\begin{table}[H]
\centering
\caption{Test \#2: Reservieren eines Rentables }
\label{my-label}
\begin{tabular}{@{}L{1.6cm}L{11cm}@{}}
\toprule
\textbf{Test \#2}
& Reservieren eines Rentables
\\ \midrule
\textbf{Ablauf}
& 
\begin{enumerate}
    \item Reserviere ein Rentable für 2 Minuten
    \item Überprüfe, ob die Reservations erfolgreich erscheint.
    \item Überprüfe, ob nach eintreten der Reservationszeit die Buttons Lock, Unlock erscheinen.
    \item Überprüfe, ob nach Aublauf der Reservation, die Buttons wieder verschwinden
\end{enumerate}
\\ \midrule
\textbf{Ergebnis}
& Die Reservation läuft erfolgreich durch. Beim Anbruch der Reservationsdauer sind Unlock und Lock ersichtlich. Nach Ablauf von ca 2. Minuten verschwinden die Buttons wieder. 
\\ \bottomrule
\end{tabular}
\end{table}


\begin{table}[H]
\centering
\caption{Test \#3: Fehleranzeige bei falscher Reservation}
\label{my-label}
\begin{tabular}{@{}L{1.6cm}L{11cm}@{}}
\toprule
\textbf{Test \#3}
& Fehleranzeige bei falscher Reservation
\\ \midrule
\textbf{Ablauf}
& 
\begin{enumerate}
    \item Reserviere ein Rentable in der Vergangenheit
    \item Überprüfe, ob eine entsprechende Fehlermeldung erscheint
\end{enumerate}
\\ \midrule
\textbf{Ergebnis}
& Das Rentable kann nicht reserviert werden, da eine Reservation in der Vergangenheit nicht möglich ist. 
\\ \bottomrule
\end{tabular}
\end{table}

\begin{table}[H]
\centering
\caption{Test \#4: Fehleranzeige bei Reservation mit Account ohne Ether}
\label{my-label}
\begin{tabular}{@{}L{1.6cm}L{11cm}@{}}
\toprule
\textbf{Test \#4}
& Fehleranzeige bei falscher Reservation
\\ \midrule
\textbf{Ablauf}
& 
\begin{enumerate}
    \item Wechsle den Account zu einem Account der 0 Ether besitzt.
    \item Reserviere ein Rentable in der Zukunft.
    \item Überprüfe, ob eine entsprechende Fehlermeldung erscheint
\end{enumerate}
\\ \midrule
\textbf{Ergebnis}
& Das Rentable kann nicht reserviert werden, da der Account nicht genügend Ether hat.
\\ \bottomrule
\end{tabular}
\end{table}
\subsection{Systemtests}
Die Systemtests testen das Gesamtsystem ausgehend vom Benutzer. Mit diesen Tests ist sichergestellt, dass der Demonstrator funktioniert.

\subsubsection{Voraussetzungen}
Für die folgenden Systemtests wird vorausgesetzt, dass das komplette System installiert und konfiguriert ist. Der Benutzer (Tester) besitzt ein Android Mobile Telefon und kann Applikationen aus dem Internet installieren.

Das Mobile Telefon muss mit dem Demonstrator WLAN verbunden sein.
Für jeden erstellten Account wird Ether vorausgesetzt.


\begin{table}[H]
\centering
\caption{Test \#1: Installieren des Mobile Apps}
\label{my-label}
\begin{tabular}{@{}L{1.6cm}L{11cm}@{}}
\toprule
\textbf{Test \#1}
& Installieren des Mobile Apps
\\ \midrule
\textbf{Ablauf}
& 
\begin{enumerate}
    \item Die Android-App via https://lokkit.io/download installieren
    \item Die Android-App starten (Benutzer muss erstellt werden)
    \item Einloggen des erstellten Benutzers
\end{enumerate}
\\ \midrule
\textbf{Ergebnis}
& Die Android-App ist erfolgreich auf dem Mobile Telefon installiert und kann gestartet werden. Der erstellte Benutzer kann eingeloggt werden.
\\ \bottomrule
\end{tabular}
\end{table}


\begin{table}[H]
\centering
\caption{Test \#2: Reservieren eines Rentables}
\label{my-label}
\begin{tabular}{@{}L{1.6cm}L{11cm}@{}}
\toprule
\textbf{Test \#2}
& Reservieren eines Rentables
\\ \midrule
\textbf{Ablauf}
& 
\begin{enumerate}
    \item Das Rentable über den QR-Code einscannen
    \item Das Rentable reservieren
    \item Ausführen von Lock und Unlock während der Reservationszeit
    \item Rückgabe des Rentables innerhalb der Reservationszeit
\end{enumerate}
\\ \midrule
\textbf{Ergebnis}
& Das Rentable konnte erfolgreich hinzugefügt werden. Die Reservation lief erfolgreich durch. Unlock und Lock haben entsprechend die Türe ge- und entsperrt. Das Rentable konnte erfolgreich zurückgeschoben werden.
\\ \bottomrule
\end{tabular}
\end{table}

\subsection{Testdruchführung am 04.06.2017}

\begin{table}[H]
\centering
\caption{Unittests des Smart Contracts Rentable}
\label{my-label}
\begin{tabular}{@{}L{4.5cm}L{4cm}L{4cm}@{}}
\toprule
Tests &
Tests erfolgreich & 
Tests fehlgeschlagen 
\\ \midrule
7 Unittests 
& 7
& 0
\\ \midrule
3 Integrationstests  
& 
& 
\\ \midrule
2 Systemtests
& 2
& 0
\\ \bottomrule
\end{tabular}
\end{table}

TODO: add all tests