\subsection{Integrationstests}
Die Integrationstests testen die einzelnen Komponenten gegen die Blockchain. 

\subsubsection{Doorman}

\paragraph{Voraussetzungen}

Um die Integrationstests von Doorman auszuführen muss folgendes Setup durchgeführt werden:

\begin{itemize}
    \item Auschecken der Doorman Resourcen
    \item Installieren der Abhängigkeiten
    \item Starten des Ethereum Nodes mittels Docker
    \item Veröffentlichen des Smart Contracts via truffle 
\end{itemize}

Danach können die Testfälle durchgespielt werden.
    
\begin{table}[H]
\centering
\caption{Test \#1: Starten von Doorman}
\label{my-label}
\begin{tabular}{@{}L{1.6cm}L{11cm}@{}}
\toprule
\textbf{Test \#1}
& Starten von Doorman mit Referenz zum veröffentlichten Contract
\\ \midrule
\textbf{Ablauf}
& 
\begin{enumerate}
    \item Eintragen der Addresse des veröffentlichten Contracts im Config file von Doorman
    \item Starten des Doormans
\end{enumerate}
\\ \midrule
\textbf{Ergebnis}
& Doorman startet und gibt Details zu allen Rentables aus, auf deren Whispers gehört werden soll.

\\ \bottomrule
\end{tabular}
\end{table}

\begin{table}[H]
\centering
\caption{Test \#2: Doorman emfpängt Lock/Unlock-Befehl vom aktuellen Mieter}
\label{my-label}
\begin{tabular}{@{}L{1.6cm}L{11cm}@{}}
\toprule
\textbf{Test \#2}
& Doorman emfpängt Lock/Unlock-Befehl vom aktuellen Mieter
\\ \midrule
\textbf{Ablauf}
& 
\begin{enumerate}
    \item Mittels Truffle das Rentable für 5 min mieten
    \item Danach den Lock-Befehl ausführen
    \item Danach den Unlock-Befehl ausführen
\end{enumerate}
\\ \midrule
\textbf{Ergebnis}
& Doorman startet und gibt eine Erfolgsmeldung für das jeweilige Rentable aus.

\\ \bottomrule
\end{tabular}
\end{table}

\begin{table}[H]
\centering
\caption{Test \#3: Doorman emfpängt Lock/Unlock-Befehl von unbekanntem Account}
\label{my-label}
\begin{tabular}{@{}L{1.6cm}L{11cm}@{}}
\toprule
\textbf{Test \#3}
& Doorman emfpängt Lock/Unlock-Befehl von unbekanntem Account
\\ \midrule
\textbf{Ablauf}
& 
\begin{enumerate}
    \item Neuen Account auf der Node erstellen
    \item Lock Befehl signiert mit dem neuen Account absenden
    \item Unlock Befehl signiert mit dem neuen Account absenden
\end{enumerate}
\\ \midrule
\textbf{Ergebnis}
& Doorman startet und gibt eine Fehlermeldung für den unbekannten account aus.

\\ \bottomrule
\end{tabular}
\end{table}

\subsubsection{Webapp}

\paragraph{Voraussetzungen}

Um die Integrationstests von der Webapp auszuführen muss folgendes Setup durchgeführt werden:

\begin{itemize}
    \item Auschecken der Webapp Resourcen
    \item Installieren der Abhängigkeiten
    \item Starten des Ethereum Nodes mittels Docker
    \item Starten der Webapp
    \item Veröffentlichen des Smart Contracts via truffle 
\end{itemize}

Danach können die Testfälle durchgespielt werden.


\begin{table}[H]
\centering
\caption{Test \#1: Hinzufügen eines Rentables }
\label{my-label}
\begin{tabular}{@{}L{1.6cm}L{11cm}@{}}
\toprule
\textbf{Test \#1}
& Hinzufügen eines Rentables
\\ \midrule
\textbf{Ablauf}
& 
\begin{enumerate}
    \item Füge das Rentable hinzu
    \item Schliesse den Browser und starte erneut
    \item Überprüfe, ob das Rentable noch hinzugefügt ist
\end{enumerate}
\\ \midrule
\textbf{Ergebnis}
& Das Rentable ist immer noch hinzugefügt.
\\ \bottomrule
\end{tabular}
\end{table}


\begin{table}[H]
\centering
\caption{Test \#2: Reservieren eines Rentables }
\label{my-label}
\begin{tabular}{@{}L{1.6cm}L{11cm}@{}}
\toprule
\textbf{Test \#2}
& Reservieren eines Rentables
\\ \midrule
\textbf{Ablauf}
& 
\begin{enumerate}
    \item Reserviere ein Rentable für 2 Minuten
    \item Überprüfe, ob die Reservations erfolgreich erscheint.
    \item Überprüfe, ob nach eintreten der Reservationszeit die Buttons Lock, Unlock erscheinen.
    \item Überprüfe, ob nach Aublauf der Reservation, die Buttons wieder verschwinden
\end{enumerate}
\\ \midrule
\textbf{Ergebnis}
& Die Reservation läuft erfolgreich durch. Beim Anbruch der Reservationsdauer sind Unlock und Lock ersichtlich. Nach Ablauf von ca 2. Minuten verschwinden die Buttons wieder. 
\\ \bottomrule
\end{tabular}
\end{table}


\begin{table}[H]
\centering
\caption{Test \#3: Fehleranzeige bei falscher Reservation}
\label{my-label}
\begin{tabular}{@{}L{1.6cm}L{11cm}@{}}
\toprule
\textbf{Test \#3}
& Fehleranzeige bei falscher Reservation
\\ \midrule
\textbf{Ablauf}
& 
\begin{enumerate}
    \item Reserviere ein Rentable in der Vergangenheit
    \item Überprüfe, ob eine entsprechende Fehlermeldung erscheint
\end{enumerate}
\\ \midrule
\textbf{Ergebnis}
& Das Rentable kann nicht reserviert werden, da eine Reservation in der Vergangenheit nicht möglich ist. 
\\ \bottomrule
\end{tabular}
\end{table}

\begin{table}[H]
\centering
\caption{Test \#4: Fehleranzeige bei Reservation mit Account ohne Ether}
\label{my-label}
\begin{tabular}{@{}L{1.6cm}L{11cm}@{}}
\toprule
\textbf{Test \#4}
& Fehleranzeige bei falscher Reservation
\\ \midrule
\textbf{Ablauf}
& 
\begin{enumerate}
    \item Wechsle den Account zu einem Account der 0 Ether besitzt.
    \item Reserviere ein Rentable in der Zukunft.
    \item Überprüfe, ob eine entsprechende Fehlermeldung erscheint
\end{enumerate}
\\ \midrule
\textbf{Ergebnis}
& Das Rentable kann nicht reserviert werden, da der Account nicht genügend Ether hat.
\\ \bottomrule
\end{tabular}
\end{table}