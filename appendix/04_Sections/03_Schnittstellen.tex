\section{Interne Schnittstellen}
In diesem Abschnitt werden die internen Schnittstellen des Lokkit Systems beschrieben.

\subsection{Real-Time Kommunikationsschnittstelle}
\label{sys_subsec:Real_Time_Kommunikationsschnittstelle}
Diese Schnittstelle dient der Echtzeitkommunikation zwischen dem Webapp (vgl. \ref{sys_subsec:Webapp}) und dem Doorman (vgl. \ref{sys_subsec:Doorman}). Sie bedient sich kryptografischer Funktionen aus der Web3 Library, um die Validität des Absenders zu validieren.

\subsubsection{Aufbau der Nachricht}
Für das gesamte von Whisper v5 versendete Packet wird der Ausdruck \emph{Envelope} verwendet. Die darin enthaltene textbasierte Nachricht, die einen validen JSON-String beinhaltet, wird als \emph{Payload} bezeichnet. Da Whisper v5 nur textbasierte Payloads unterstützt, wurde eine Nachricht im \acrshort{JSON} Format definiert, die eine strukturierte Möglichkeit zur Datenübermittlung ermöglicht. Diese wird übermittelt, indem sie vom Sender zu einer String-Repräsentation umgewandelt wird und nach Erhalt beim Empfänger wiederum zu einem JSON geparst wird. Das definierte Objekt besitzt zwei Attribute: \emph{digest} (vgl. \ref{sys_para:Digest}), nachfolgend Diges und \emph{message} (vgl. \ref{sys_para:Message}), nachfolgend Message Objekt.

\begin{lstlisting}[language=json,caption={Beispiel einer Lokkit JSON Nachricht, deren \emph{digest} vom account \emph{0x55047206d03afef0d79bb2d90710bf9f23737860} erstellt wurde.}]
{
  digest: "0x06351aa8220219b621d7c2c5340110d3be147b92ef8035469df996762e021f512f4f8  30dee0ed1d59998b4bafaa3abb589668d940bdce867f7e6d83315a454af1b",
  message: {
    command: "open",
    key: "435712b75b3546c06ed5e9ab19fbc309c5f2e0a57701e56e39f43f449b6e72c9",
    rentableAddress: "0xbac41871cf121f6c02d345af52c82c83759a2e3b"
  }
}
\end{lstlisting}

\paragraph{Message}
\label{sys_para:Message}
Das Message Objekt innerhalb der Payload enthält drei Attribute: \emph{command}, \emph{rentableAddress} und \emph{key}. Funktional relevant für den Empfänger sind dabei nur command und rentableAddress. Das key Attribut ist eine zusätzliche Sicherheit, um Replay-Angriffe mit demselben Packet zu verhindern, bei welchen dasselbe Paket von einem anderen Sender versendet wird.
\subparagraph{Command}
\label{sys_subpara:Command}
Dieses Feld teilt den Befehl mit, den das mietbare Objekt ausführen soll. Abhängig von der jeweiligen Implementation des IoT Geräts kann dieses Feld andere Werte beinhalten. Im Beispiel von Lokkit werden \emph{lock} und \emph{unlock} für Schliessfächer geschickt. Weitere Befehle sind denkbar, die ebenfalls den Doorman als Empfänger verwenden.
\subparagraph{Rentable Address}
Damit der Empfänger weiss, welches mietbare Objekt von dem Befehl angesteuert werden soll, wird die gesamte Adresse mitgegeben. Alternativ könnte dieses Feld weggelassen werden und über das Topic der Nachricht bestimmt werden.

\subparagraph{Key}
\label{sys_para:Key}
Dieses Feld beinhaltet den öffentlichen Schlüssel eines asymmetrischen Schlüsselpaars des Senders. Derselbe Schlüssel muss zum signieren des Envelopes verwendet werden\footnote{Diese Signatur ist optional bei Whisper, wird jedoch für Lokkit Envelopes vorgeschrieben}. Der öffentliche Schlüssel des für die Signatur verwendeten Schlüsselpaars wird ebenfalls im Envelope als Feld \emph{sig} mitgeschickt. Bei einer validen Nachricht sind also das Feld \emph{key} des Message Objektes und das Feld \emph{sig} des Envelopes identisch\footnote{Für weitere Informationen, vgl. \ref{para:Replay_Attack}}.

\paragraph{Digest}
\label{sys_para:Digest}
Um den Digest eines Message Objekts zu erzeugen wird die Funktion \emph{web3.eth.sign} verwendet. Dieser Funktion wird als einziger Parameter die Stringrepräsentation des Message Objektes übergegeben. Bei entsperrtem Account liefert die Funktion einen 65 Byte Hash, den Digest, zurück. Dabei ist zu beachten, dass derselbe Account entsperrt ist, der auch das Rentable gemietet hat, an das diese Nachricht gesendet werden soll. Um herauszufinden, welche Ethereum Adresse den Digest des Message Objektes erzeugt hat, wird die Funktion \emph{web3.personal.ecRecover} verwendet. Diese nimmt als Parameter die Stringrepräsentation des Message Objektes und als zweiten Parameter den Digest. Als Rückgabewert liefert diese dann die Adresse, die die Signatur erstellt hat.

Wichtig dabei ist, dass die über Whisper erhaltenen JSON Objekte jeweils die Attribute in alphabetisch sortierter Reihenfolge beinhalten und über die javascript Funktion JSON.stringify serialisiert werden, bevor sie den kryptographischen Funtionen übergeben werden. Grund dafür ist, dass besagte sign Funktion eine Zeichenkette als Arugment nimmt und nicht ein Objekt. Die Verwendung von JSON.stringify versichert, dass der Eingabewert stets gleich strukturiert ist.

\begin{lstlisting}[language=javascript,caption={beispiel generate digest und ecRecover}]
// Sender
var message = { 'command': command, 'rentableAddress': rentableAddress, 'key': publicKey };
var messageBytes = web3.fromAscii(JSON.stringify(message));
var digest = web3.eth.sign(adi, messageBytes);

// Empfänger
var payload = JSON.parse(web3.toAscii(envelope.payload));
var signer = web3.personal.ecRecover(web3.fromAscii(JSON.stringify(payload.message)), payload.digest);
\end{lstlisting}
Dabei wird der Empfänger als Resultat von \emph{ecRecover} die Adresse erhalten, die beim Sender entsperrt war, als \emph{web3.eth.sign} aufgerufen wurde.

\subsubsection{Verteilung}
\label{sys_subsubsec:Verteilung}
Als key attribut des Envelopes wird ein symmetrischer Schlüssel mit dem Passwort \emph{lokkit} erstellt. Als Topic werden die ersten 4 Bytes der sha3 Checksumme der Adresse des Mietbaren Objektes mitgeschickt. Basierend auf diesen beiden Eigenschaften kann der Doorman (vgl. \ref{subsec:Doorman}) die für die angeschlossenen Geräte relevanten Nachrichten filtern.

\begin{lstlisting}[language=javascript,caption={Beispiel einer Whisper v5 Nachricht}]
var msg = "hello world";
var key = web3.shh.gensymkey();
var topic = 0xFFFF;
shh.post({key:key, topic:topic, payload:web3.fromAscii(msg)});
\end{lstlisting}


\begin{lstlisting}[language=javascript,caption={Beispiel eines Whisper v5 Filters}]
var shhPw = "lokkit";
var key = shh.addSymmetricKeyFromPassword(shhPw);
var topic = web3.sha3(rentableAddress).substr(0, 10);
var filterId = web3.shh.subscribe({type: 'sym', topics: [topic], key: key});
web3.shh.getFilterUpdates(filterId).forEach(function(err, result) {
    // result ist der envelope.
    console.log(result.sig);
    console.log(result.payload);
}
\end{lstlisting}
