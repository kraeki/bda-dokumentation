\chapter{Installation \& Deployment}
\label{cha:Installation_Deployment}

In den folgenden Abschnitten wird erklärt wie das System aufgesetzt wird. Jede Komponente wird aus den Sourcen gebildet und dann auf den Raspberries installiert. Dazu wird zuerst erklärt wie die Raspberries in Betrieb genommen werden.

\section{Voraussetzungen für das Deployment}
Die nachfolgende Anleitung wurde unter einem aktuellen GNU/Linux System erfolgreich getestet. Grundsätzlich sind andere Systeme wie Windows oder MAC jedoch nicht ausgeschlossen und können für das Deployment ebenfalls verwendet werden. Dies jedoch auf eigene Verantwortung.

Konfigurationsübersicht zu diesem Zeitpunkt

\begin{table}[H]
\centering
\caption{Systemvorausetzungen um das Deployment durchzuführen}
\label{my-label}
\begin{tabular}{@{}lll@{}}
\toprule
Komponente 
& Version               
& Beschreibung                           
\\ \midrule
GNU/Linux  
& Ubuntu 16.04 (Xenial) 
& Basis System für das Deployment        
\\ \midrule
nodejs     
& 7.10.0                
& Voraussetzung für Webapp               
\\ \midrule
npm        
& 4.2.0                 
& Voraussetzung für Webapp               
\\ \midrule
git        
&  2.7.4                
& Zum Auschecken der Sourcen               
\\ \midrule
truffle    
& 3.2.4                 
& Für das Deployment der Smart Contracts 
\\ \midrule
geth       
& 1.6.2                 
& Ethereum Node                          
\\ \midrule
doorman    
&                       
&                           
\\ \midrule
Webapp     
&                       
&                                        
\\ \bottomrule
\end{tabular}
\end{table}

\paragraph{Herunterladen der Resourcen}

Die Lokkit Sourcen können wie folgt heruntergeladen werden:

\begin{lstlisting}[language=bash]
$ git clone https://github.com/lokkit/lokkit --recursive
\end{lstlisting}

\section{Aufsetzten der Raspberry Pi's}
Für jedes der drei Raspberry Pi's wurde ein eigenes Image erstellt. Die Images basieren auf dem offiziellen \emph{Raspbian Jessie Lite} Image \footnote{\url{https://www.raspberrypi.org/downloads/raspbian/}}.

\paragraph{}
\begin{description}
    \item[Lokkit Image 1] Dieses Image enthält die Konfiguration für den WLAN Hotspot und einen DNS/DHCP Server. Die anderen Raspberries verbinden sich zu diesem.
    \item[Lokkit Image 2] Dieses Image ist so konfiguriert, dass es sich mit dem HotSpot des Master Images verbindet.
    \item[Lokkit Image 3] Dieses Image ist bis auf den Hostnamen identisch mit dem Lokkit Image 2
\end{description}

\paragraph{}
Alle drei Images können z.B. mit \emph{dd} unter GNU/Linux auf eine MicroSD geflashed werden. 

\begin{lstlisting}[language=bash,caption={Beispiel \emph{dd} unter GNU/Linux}]
$ dd if=rasp-master.img of=/dev/sdd
$ dd if=rasp-node1.img of=/dev/sdd
$ dd if=rasp-node2.img of=/dev/sdd
\end{lstlisting}

Danach können die Raspberries mit dem Image gestartet werden und nach spätestens 5 Minuten sollte ein Hotspot \emph{lokkit} ersichtlich sein. Die Zugangsdaten für das WLAN können der Tabelle \ref{tbl:wlan_zugang} entnommen werden.  

\begin{table}[]
\centering
\caption{WLAN Zugangsdaten}
\label{tbl:wlan_zugang}
\begin{tabular}{@{}ll@{}}
\toprule
SSID    & Passwort \\ \midrule 
lokkit  & trust-no-one \\ \bottomrule
\end{tabular}
\end{table}


Nach erfolgreichem Verbinden mit dem \emph{lokkit} Hotspot ist es nun möglich auf die Raspberries über \emph{SSH} zuzugreifen. Im untenstehenden Beispiel wird eine Verbindung zum Master-RaspberryPi hergestellt. 

\begin{lstlisting}[language=bash,caption{SSH Verbindung mit Master}]
$ ssh master.lokkit.io -l pi
# oder auch
$ ssh 192.168.0.1 -l pi
\end{lstlisting}

Die Benutzer und Passwörter der Raspberry Pi's sind der Tabelle \ref{tbl:raspberry_zugang} zu entnehmen.

\begin{table}[]
\centering
\caption{SSH Zugang}
\label{tbl:raspberry_zugang}
\begin{tabular}{@{}lll@{}}
\toprule
Host              & Benutzer   & Passwort \\ \midrule
master.lokkit.io  & pi         & TrNoOnJuLo \\ \midrule
node1.lokkit.io   & pi         & TrNoOnJuLo \\ \midrule
node2.lokkit.io   & pi         & TrNoOnJuLo \\ \bottomrule
\end{tabular}
\end{table}


\paragraph{Netzwerk} 
Der Hotspot verwendet das \emph{192.168.0.0/24} Netzwerk. Falls Probleme mit der DNS-Auflösung auftreten, kann das Master Raspberry Pi auch direkt mit der IP \emph{192.168.0.1} angesprochen werden.

\paragraph{}
Nachdem nun die Raspberry Pi's aufgesetzt und gestartet sind, können die Softwarekomponenten installiert werden. Die nächsten Kapitel zeigen, wie dies geht.








\section{Aufsetzen der Private Chain}
Die Private Chain besteht aus Ethereum Nodes (geth), welche so konfiguriert sind, dass sie ein eigenes neues Netzwerk erstellen und nicht ins eigentliche Ethereum Netzwerk (Homestead) verbinden. Der \emph{geth}\footnote{\url{https://geth.ethereum.org/}} Client hat ein breites Interface und kann über die Kommandozeile konfiguriert werden. Um die Installation und die Konfiguration der Lokkit Private Chain zu vereinfachen wurde ein Debian Paket kreiert, welches auf dem Raspberry-Pi installiert werden kann und, dass den \emph{geth} Client mit allen benötigten Parameter aufsetzt und initialisiert. Nachfolgend wird gezeigt, wie das Debian Paket erstellt und auf den Raspberry Pi's installiert wird. 

\subsection{Builden der Debian Pakete}
Zuerst muss ins Repository \emph{chain} gewechselt werden. Dann können die Pakete folgendermaßen gebuildet werden:

\begin{lstlisting}[language=bash]
$ cd chain # ins Verzeichnis wechseln
$ debuild -us -uc # erstellt 2 Pakete ../lokkit-chain*.deb
\end{lstlisting}

Das Paket \emph{lokkit-chain-no-mine} kann dann auf die Raspberry Pi's kopiert und installiert werden. Das nächste Code Snippet muss für jedes Raspberry Pi durchgeführt werden.

\begin{lstlisting}[language=bash]
$ scp ../lokkit-chain-no-mine*.deb pi@master.lokkit.io:

$ ssh pi@master.lokkit.io
pi@master:~ $ sudo su
Passwort: 
pi@master:~ $ dpkg -i lokkit-chain-no-mine*.deb
\end{lstlisting}

\paragraph{}
Somit ist die Private Chain auf den Raspberry Pi's installiert und lauffähig. Sie ist jedoch noch leer, sprich es gibt noch keinen Miner, der Blöcke mined. Analog zur Installation der Raspberry Pi's muss nun noch die Miner-Hardware installiert werden. Dazu wird das Paket \emph{lokkit-chain-mine} verwendet. Das Endresultat ist eine Private Chain bestehend aus fünf Nodes. Zwei Nodes auf der Miner-Hardware und jeweils ein Node pro Raspberry Pi.

\section{Smart Contracts installieren}
Um die drei Rentable Contracts auf der Blockchain zu veröffentlichen, muss in das Repository \emph{contracts} gewechselt werden. Anschliessend kann mittels \emph{truffle} die Contracts deployed werden. 

\paragraph{}
Damit \emph{truffle} die Contracts veröffentlichen kann, muss der erste Account auf dem Master-Rapsberry entsperrt werden. Zu diesem Zweck muss auf dem Raspberry Pi eingeloggt werden und mittels \emph{geth attach} der erste Account entsprerren. Das untenstehende Code Snippet illustriert das Vorgehen. 

\begin{lstlisting}[language=bash,label={lst:rentable_deployment},caption={Deployment der Contracts}]
$ ssh master.lokkit.io -l pi 
pi@master:~ $ geth attach
personal.unlockAccount(personal.listAccounts[0])
pi@master:~ $ exit
$ exit

# Nun können die Contracts mit truffle veröffentlicht werden
$ cd contracts # ins Verzeichnis wechseln
$ truffle deploy --network lokkit
Using network 'lokkit'.

Running migration: 1_initial_migration.js
  Replacing Migrations...
  Migrations: 0x9624922904e8907373b23dfbe7723be094abf5d6
Saving successful migration to network...
Saving artifacts...
Running migration: 2_deploy_contracts.js
  Replacing Rentable...
  Rentable: 0x58b5e51386b46d018dcd0a3db91a6c69fed20ea8
  Replacing Rentable...
  Rentable: 0xde93b2965af6a49161f597604c600af9ea07883a
  Replacing Rentable...
  Rentable: 0x75105e510adf9fa0b1b1a5d35f9d8594ad36d8ed
Saving successful migration to network...
\end{lstlisting}

\noindent 
Die Adressen der veröffentlichten Rentables sind im Konsolenoutput von \emph{truffle} ersichtlich (vgl. \ref{lst:rentable_deployment}). Diese Adressen können nun mittels QrCode Generator\footnote{Z.B.: \url{http://goqr.me/de/}} gespeichert und ausgedruckt werden. Der Demonstrator hat für jedes Schliessfach auf der Vorderseite Platz für einen solchen.

\section{Webapp}
Die Webapp wird ebenfalls per Debian Paket auf dem Master Raspberry Pi installiert. Nach erfolgreicher Installation ist sie unter \url{https://webapp.lokkit.io} aufrufbar. Um mit der Installation zu starten muss ins Repository \emph{webapp} gewechselt werden. Anschliessend kann das Debian Paket gebuildet und auf das Raspberry Pi kopiert und installiert werden. Der folgende Code illustriert das Vorgehen:

\begin{lstlisting}[language=bash]
$ cd webapp # ins Verzeichnis wechseln
$ debuild -us -uc # erstellt 1 Paket ../lokkit-webapp*.deb
$ scp ../lokkit-webapp*.deb pi@master.lokkit.io:
$ ssh pi@master.lokkit.io
pi@master:~ $ dpkg -i lokkit-webapp*.deb
\end{lstlisting}

Zum Testen kann jetzt per Web Browser auf \url{https://webapp.lokkit.io} zugegriffen werden.

\section{Doorman}
Doorman muss auf allen drei Raspberry Pi's installiert und dann unterschiedlich konfiguriert werden. Die Installation erfolgt ebenfalls durch ein Debian Paket.

\begin{lstlisting}[language=bash]
$ cd doorman # ins Verzeichnis wechseln
$ debuild -us -uc # erstellt 1 Paket ../lokkit-doorman*.deb
$ scp ../lokkit-doorman*.deb pi@master.lokkit.io:
$ ssh pi@master.lokkit.io
pi@master:~ $ dpkg -i lokkit-doorman*.deb
\end{lstlisting}

Nach erfolgreicher Installation von Doorman, muss dieser jeweils mit der, für das Schliessfach angepassten, Konfiguration gestartet werden. Die Konfiguration enthält jeweils die Adresse des entsprechenden Rentables und verweist auf ein Script, das das Schloss ansprechen kann. Beim Locker 2 und 3 sind die Schlösser identisch, und somit auch das Script. 

\paragraph{}
Die Konfiguration von Doorman liegt unter \emph{/etc/lokkit/doorman.yaml} und muss folgenden Inhalt haben:

\begin{lstlisting}[language=yaml,caption={Doorman-Konfiguration für Locker 2 und 3}]
doorman:
    node_ip: 127.0.0.1
    node_rpc_port: 8545
    rentable_addresses:
        - "0x58b5e51386b46d018dcd0a3db91a6c69fed20ea8" # Achtung: Korrekte Adresse eintragen
    symmetric_key_password: lokkit
    script: /opt/lokk.sh
\end{lstlisting} 

\paragraph{} 
Für Locker 1 und 2 muss noch die Datei \emph{/opt/lokk.sh} erstellt werden, welches den GPIO ansteuert und somit das Schloss öffnen kann.

\begin{lstlisting}[language=bash,caption={Script für Locker 2 und 3, das von Doorman aufgerufen wird}]
#!/bin/sh

port=17 # Verwendeter GPIO Port
echo Initializing port $port
echo $port > /sys/class/gpio/export
echo "out" > /sys/class/gpio/gpio${port}/direction
echo Initialization done

# Öffnen des Schlosses
echo "1" > /sys/class/gpio/gpio${port}/value

# Warten von 3 Sekunden
sleep 3

# Schliessen
echo "0" > /sys/class/gpio/gpio${port}/value
\end{lstlisting} 

\paragraph{}
Auf dem Locker 1 muss ein anderes Script hinterleget werden, welches das Nuki Schloss anprechen kann. Zuerst muss aber das Script auf dem Raspbery-Master (Locker 1) installiert werden. Dazu wird das Repository \emph{nuki} direkt auf das Master Raspberry kopiert. Anschliessend wird in der Doorman Konfiguration das Script auf /opt/nuki/lokk.sh geändert.

\begin{lstlisting}[language=bash]
$ scp nuki pi@master.lokkit.io:
$ ssh pi@master.lokkit.io
pi@master:~ $ sudo mv nuki /opt
\end{lstlisting}


\section{Android App}
Die Android App ist das letzte Stück dieser Kette. Sie kann unter \url{https://lokkit.io/download} heruntergeladen und installiert werden.