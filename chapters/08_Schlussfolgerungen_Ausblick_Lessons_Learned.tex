\chapter{Schlussfolgerung}
\label{cha:Schlussfolgerung}

\section{Ausblick}
\label{sec:Ausblick}
Blockchain für realtime Anwendungen wahrscheinlich nicht geeignet?
\par
Anonymität kann auf gefährlich sein, da man nicht weiss, von wem man eine Dienstleistung kauft. -> Wie im Darknet kann es sein, dass eine Dienstleistung bezahlt wird, aber nicht erhalten wird.
Gibt bei zentralisierten Angeboten (wie bspw. isoliert an einem Bahnhof oder in einer Hochschule) keinen ersichtlichen Mehrwert gegenüber einer ebenfalls zentralisierten Anwendung. Einige Nachteile wie bspw. Latenz, Verfügbarkeit oder Vertrauenswürdigkeit des Angebots sind sogar ersichtlich.

\subsection{Weitere Gedanken}
Bei Recherche gefundene Konzepte beinhalten auch bspw. Elektroautos an einem Lichtsignal zu tanken. Ist nicht wirklich machbar, da 12 Sekunden Blockcycle zu lange sind, um zu garantieren, dass eine Leistung bezogen werden kann (Transaktion ist erst sicher, sobald der block gemined ist...)



\section{Lessons Learned}
\label{sec:Lessons_Learned}

\subsection{Dominik Hirzel}
Die Abschlussarbeit war natürlich sehr spannend und sexy....

\subsection{Andreas Schmid}
Ich bin de Andi. Zum glück hani mich für s Studium ade HSLU entschiede, susch hätti mir no müsse müe gä...
