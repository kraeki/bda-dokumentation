\chapter{Schlussfolgerung}
\label{cha:Schlussfolgerung}
\begin{itemize}
    \item \textbf{Wie beurteilen Sie die Blockchain-Technologie? Generell? im Bezug auf IoT? Gibt's offensichtliche Hindernisse?}
    \item \textbf{Was wuerden Sie im Projekt anders machen?}
    \item \textbf{Was hat am meisten Zeit gekostet?}
    
    \item \textbf{Sehen Sie Zukunft in dieser technologie?}
    \item \textbf{}
\end{itemize}

\section{Ausblick}
\label{sec:Ausblick}
\begin{itemize}
    \item \textbf{Ist es denkbar den Demonstrator produktive einzusetzen? Ist es ein reales Produkt, dass eine Firma einsetzen wollen wuerde?}
    \item \textbf{Was wird sich von der Technologie her noch aendern?}
    \item \textbf{Was sind die Challenges fuer die Zukunft?}
\end{itemize}
Blockchain für realtime Anwendungen wahrscheinlich nicht geeignet?
\par
Anonymität kann auf gefährlich sein, da man nicht weiss, von wem man eine Dienstleistung kauft. -> Wie im Darknet kann es sein, dass eine Dienstleistung bezahlt wird, aber nicht erhalten wird.
Gibt bei zentralisierten Angeboten (wie bspw. isoliert an einem Bahnhof oder in einer Hochschule) keinen ersichtlichen Mehrwert gegenüber einer ebenfalls zentralisierten Anwendung. Einige Nachteile wie bspw. Latenz, Verfügbarkeit oder Vertrauenswürdigkeit des Angebots sind sogar ersichtlich.

\subsection{Weitere Gedanken}
Bei Recherche gefundene Konzepte beinhalten auch bspw. Elektroautos an einem Lichtsignal zu tanken. Ist nicht wirklich machbar, da 12 Sekunden Blockcycle zu lange sind, um zu garantieren, dass eine Leistung bezogen werden kann (Transaktion ist erst sicher, sobald der block gemined ist...)



\section{Lessons Learned}
\label{sec:Lessons_Learned}

\subsection{Dominik Hirzel}
Das war das interessanteste und lehrreichste Projekt, an dem ich im schulischen und professionellen Umfeld je gearbeitet habe. Nie wollte ich mit einem meiner Kommilitonen tauschen. Diese Aussage dient nicht dazu, eine bessere Note ergaunern zu wollen, sondern soll vielmehr meine Euphorie, meinen Arbeits- und Lernwillen in diesem Projekt ausdrücken. Einen grossen Teil dieser Motivation und Ausdauer verdanke ich auch meinem technisch wie sozial ausserordentlich versierten Partner: Andreas Schmid.

Die cutting edge Technologie der Blockchain interdisziplinär mit dem IoT Aspekt zu verbinden hat mir nicht nur schlaflose Nächte und eine Bekanntschaft mit dem Nachtwächter der HSLU beschert, sondern auch viel Freudensprünge erlaubt. Dieses Projekt hat mich in meiner Entscheidung bekräftigt, die richtige Ausbildung und das richtige Studium gewählt zu haben.

Speziell die Zusammenarbeit mit den open source Projekten Ethereum und Status-IM hat mir gefallen. Die Tatsache, dass wir ihnen in ihrem Design und der Implementation Fehler gefunden haben und diese mitteilen konnten (vgl. \ref{subsec:Open_Source}) bestätigt auch die Intensivität und unser Engagement in dieser Arbeit.

Die unterschiedlichen entworfenen Softwarekomponenten und das Zusammenspiel dieser, sowie das Design der Real-Time Kommunikationsschnittstelle und Berücksichtigung der Sicherheitsrelevanten Aspekte waren natürlich auch aus technischer Sicht ein Higlight dieser Arbeit. Allgemein lässt sich die Komplexität dieses Systems betonen, das wir mit klar abgegrenzten Verantwortungsbereichen entworfen und umgesetzt haben.

\subsection{Andreas Schmid}
Ich bin de Andi. Zum glück hani mich für s Studium ade HSLU entschiede, susch hätti mir no müsse müe gä...
