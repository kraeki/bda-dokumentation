\chapter{Aufgabenstellung}
\label{cha:Aufgabenstellung}

Im diesem Kapitel wird die Aufgabenstellung aufgezeigt, die im Rahmen dieser 
Arbeit ebefalls erstellt wurde. Einzig die Schlagworte \emph{Blockchain},
\emph{\acrfull{IoT}} und den Bau eines Demonstrators waren vorgegeben.

Die \emph{Blockchain}-Technologie ist zurzeit stark im Trend. Sie benutzt verteilte
und dezentrale Rechennetzwerke und bietet dadurch bessere Sicherheit und geringere 
Kosten im Vergleich zu traditionellen Methoden.

Neue Blockchains unterstützen neben Crypto-Währungen auch sogenannte \emph{Smart Contracts},
die es dem Benutzer erlauben, Programme für die Blockchain zu implementieren. Diese
Art von Programm nennt man auch \emph{\acrfull{DAPP}}.

Auch das Thema \acr{IoT} ist in aller Munde und wächst in den kommenden Jahren stark an. Gartner 
schätzt Zuwachs von 600000000000 neuen Geräten bis 2020. Dies bringt neue Anforderungen an
die Sicherheit mit - ein Aspekt, der eventuell durch die Blockchain-Technologie erfüllt
werden kann.

Im Rahmen dieser Bachelorarbeit soll ein System entwickelt werden, das den IoT und Blockchain Aspekt
umsetzt und demonstriert.

\begin{itemize}
    \item \textbf{ Wieso sind Blockchain und IoT wichtig? Gartner Hype Cycler}
    \item \textbf{ Wieso schlaegt die Schule dieses Thema vor? }
    \item \textbf{ Was soll mit dem Demonstrator erreicht werden? }
\end{itemize}

\section{Anforderungen an den Demonstrator}
\label{sec:Anforderungen an den Demonstrator}
\begin{itemize}
    \item Der Demonstrator soll auf dem Schulareal ausgestellt werden können. 
    \item Der Demonstrator soll wiederverwendbar sein.
    \item Der Aufbau des Demonstrators sowie die Inbetriebnahme muss dokumentiert sein.
\end{itemize}

\section{Konkrete Aufgabenstellung}
\label{sec:Konkrete Aufgabenstellung}
Es soll ein System entwickelt werden, um Schließfächer an Personen zu vermieten. Jedes Schließfach muss elektronisch ver- und entriegelt werden können.

Der Benutzer soll über eine Oberfläche ein freies Schließfach reservieren können. Sobald der Benutzer der aktuelle Mieter eines Faches ist, kann er das Schloss per Knopfdruck öffnen und schliessen.

Im Fokus steht die Umsetzung mit einer geeigneten Blockchain.

\section{Ziele}
\label{sec:Ziele}
\begin{itemize}
    \item \textbf{ Was sind die Ziele die durch diese Arbeit erreicht werden sollen? }
    \item \textbf{ Was ist das Ziel des Demonstrators? }
\end{itemize}

\begin{itemize}
    \item Demonstrieren der Vorteile der Blockchain
    \item \dots
\end{itemize}


\section{Erwartete Resultate}
\label{sec:Erwartete_Resultate}

Ein lauffähiger Prototyp des Schliessfachvermietsystems.