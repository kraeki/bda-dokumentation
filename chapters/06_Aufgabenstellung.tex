\chapter{Aufgabenstellung}
\label{cha:Aufgabenstellung}

Im diesem Kapitel wird die Aufgabenstellung aufgezeigt, die im Rahmen dieser 
Arbeit ebefalls erstellt wurde. Einzig die Schlagworte \emph{Blockchain},
\emph{\acrfull{IoT}} und den Bau eines Demonstrators waren vorgegeben.

Die \emph{Blockchain}-Technologie ist zurzeit stark im Trend. Sie benutzt verteilte
und dezentrale Rechennetzwerke und bietet dadurch bessere Sicherheit und geringere 
Kosten im Vergleich zu traditionellen Methoden.

Neue Blockchains unterstützen neben Crypto-Währungen auch sogenannte \emph{Smart Contracts},
die es dem Benutzer erlauben, Programme für die Blockchain zu implementieren. Diese
Art von Programm nennt man auch \emph{\acrfull{DAPP}}.

Auch das Thema \acr{IoT} ist in aller Munde und wächst in den kommenden Jahren stark an. Gartner 
schätzt Zuwachs von 600000000000 neuen Geräten bis 2020. Dies bringt neue Anforderungen an
die Sicherheit mit - ein Aspekt, der eventuell durch die Blockchain-Technologie erfüllt
werden kann.

Im Rahmen dieser Bachelorarbeit soll ein System entwickelt werden, das den IoT und Blockchain Aspekt
umsetzt und demonstriert.

\section{Anforderungen an den Demonstrator}
\label{sec:Anforderungen an den Demonstrator}
\begin{itemize}
    \item Der Demonstrator soll auf dem Schulareal ausgestellt werden können. 
    \item Der Demonstrator soll wiederverwendbar sein.
    \item Der Aufbau des Demonstrators sowie die Inbetriebnahme muss dokumentiert sein.
\end{itemize}

\section{Konkrete Aufgabenstellung}
\label{sec:Konkrete Aufgabenstellung}


\section{Ziele}
\label{sec:Ziele}

\section{Erwartete Resultate}
\label{sec:Erwartete_Resultate}

This is a reference to the glossery: \gls{maths}.

And here I am using a acronym: \acrfull{lcm}.