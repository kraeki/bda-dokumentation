\chapter{Abstract d}
\label{cha:abstract_d}

Blockchain. Kryptowährungen. Dezentralisierte Datenbank. Internet of Things. Was sich für Laien anhört wie ein Auszug aus dem Computerduden, lässt die Herzen von so manchem Informatikstudenten höher schlagen. Bei diesen Begriffen handelt es sich nicht nur um Ausdrücke zur Geltendmachung der mentalen Überlegenheit durch Kenntnis von Fachbegriffen. Betont werden vor allem die rasant wachsende Anzahl vernetzter Geräte, die zunehmende Verteilung von Computersystemen \footnote{Referenz zu verteilten Energiesystemen, selbst fahrenden Autos o.Ä.?}, die immer grösser werdende Frage der Sicherheit derselben\footnote{Stichwort Globalisierung, Vernetzung} und die zunehmende Unlust von Unternehmen einander in der heutigen globalisierten Welt zu vertrauen\footnote{Kryptowährung \& BC}.

Pioniert wurde die Blockchain Technologie durch eine unter dem Pseudonym bekannte Person Satoshi Nakamoto. Dieser legte den Grundstein für diese sogenannten verteilten Ledger und implementierte im Jahr 2007  die heute immer noch bekannteste Kryptowährung: Bitcoin. Neue Blockchain Implementationen unterstützen neben Kryptowährungen auch sogenannte \emph{Smart Contracts}. Diese erlauben es Benutzern, Programmcode zu implementieren, der in der Blockchain abgelegt wird. Dieser Code kann beliebige Daten in diese verteilte Datenbank schreiben und wieder abfragen. Dadurch, dass dieser Code, und auch die Daten, verteilt und somit von jedermann einsehbar sind, können Verträge unmissverständlich und öffentlich verfügbar in einem von Maschinen lesbaren Format definiert werden.

Programme, die diese Smart Contracts verwenden, nennt man \emph{\acrfull{DAPP}}. Diese 


\cite{BlockchainRevolution}

Auch das Thema \acrshort{IoT} ist seit mehreren Jahren in aller Munde und wächst in den kommenden Jahren stark an. Gartner schätzt einen Zuwachs von 20.4 Milliarden Geräten bis 2020. Dies bringt neue Anforderungen an die Sicherheit mit - ein Aspekt, der eventuell durch die Blockchain-Technologie erfüllt werden kann.\cite{gartner.com_iot,BlockchainRevolution}

Im Rahmen dieser Bachelorarbeit wurden basierend auf ausgiebiger Recherche erwähnte Smart Contracts entwickelt und durch 


\chapter{Abstract e}
\label{cha:abstract_e}

Abstract goes here
