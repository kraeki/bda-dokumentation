\chapter{Abstract d}
\label{cha:abstract_d}
Blockchain. Kryptowährungen. Dezentralisierte Datenbank. Internet of Things. Was sich für Laien anhört wie ein Auszug aus dem Computerduden, lässt die Herzen von so manchem Informatikstudenten höher schlagen. Bei diesen Begriffen handelt es sich nicht nur um Ausdrücke zur Geltendmachung der mentalen Überlegenheit durch Kenntnis von Fachbegriffen. Betont werden vor allem die rasant wachsende Anzahl vernetzter Geräte, die zunehmende Verteilung von Computersystemen, die immer grösser werdende Frage der Sicherheit derselben und die zunehmende Unlust von Unternehmen einander in der heutigen globalisierten Welt zu vertrauen.

Pioniert wurde die Blockchain Technologie durch eine unter dem Pseudonym Satoshi Nakamoto bekannte Person, deren echte Persönlichkeit bis heute unbekannt ist. Diese implementierte im Jahr 2008 die heute immer noch bekannteste Blockchain mit gleich heissender Kryptowährung: Bitcoin. Neuere Blockchain Implementationen unterstützen neben Kryptowährungen auch sogenannte \emph{Smart Contracts}. Diese bestehen aus Programmcode, der in der Blockchain abgelegt wird. Dieser Programmcode kann beliebige Daten in diese verteilte Datenbank schreiben und wieder abfragen. Dadurch, dass dieser Code, und auch die Daten, verteilt und somit von jedermann einsehbar sind, können Verträge unmissverständlich und öffentlich verfügbar in einem von Maschinen lesbaren Format definiert werden. Änderungen können nicht im geheimen vorgenommen werden, da die Blockchain garantiert, dass jede dezentrale Datenbank denselben Stand aufweist. Dies fördert die Transparenz und beseitigt die Notwendigkeit für eine Drittpartei, der beide Vertragsparteien vertrauen. Dadurch wird erreicht, dass beide Parteien zu ihrem besten Können auf die Erfüllung des Vertrags hin arbeiten.

Auch das Thema \acrshort{IoT} ist seit mehreren Jahren, nicht zuletzt dank dem Begriff Industrie 4.0, aller Munde. Bis 2020 schätzt Gartner einen Zuwachs von momentan 8.3 Milliarden auf 20.4 Milliarden Geräte. Dies birgt neue Anforderungen an die Sicherheit und Kommunikationsweise der Geräte mit sich - ein Aspekt, der eventuell durch die Blockchain-Technologie gelöst werden kann.\cite{gartner.com_iot,BlockchainRevolution}

Im Rahmen dieser Bachelorarbeit wurden spezifische Smart Contracts entworfen, die durch eine sogenannte \acrfull{DAPP} von Desktop Rechnern und Androidgeräten angesteuert werden können. Diese Smart Contracts wurden ebenfalls in einen IoT Demonstrator integriert, um eine physische Reaktion des Resultats eines Smart Contracts zu veranschaulichen.


\chapter{Abstract e}
\label{cha:abstract_e}

Blockchain. Crypto Currencies. Decentralized Database. Internet of Things. What a layman may mistake as an excerpt of the computer bible excites many computer science students. These words not only serve the purpose of asserting one's mental superiority through knowledge of technical terms. They also highlight the rapid growth in connected devices, the ever-increasing decentralization of computer systems, security concers of those exact systems, as well as the rising reluctance of companies to trust each other in today's globalized economy and society.

The blockchain technology was pioneered by an individual known as Satoshi Nakamoto, whose identity remains unknown to this date. In 2008 this person implemented the best-known blockchain, hosting the crypto currency of the same name: Bitcoin. More recent blockchain implementations support the creation of smart contracts on top of the crypto currency aspects. These smart contracts are written as source code and are saved to the blockchain. It can write arbitrary data to the blockchain and query that same information later. Since this code as well it's data are decentralized, and therefore visible to everyone, contracts can be written in a computerized format to be unambiguous and publicly available. Changes cannot be made in secret, because the blockchain ensures guarantees the same state on each and every machine. This promotes transparency and eliminates the need of a third party, who both parties trust in case of a dispute. This achieves that both parties work to the contract's fulfillment at their best ability.

The term \acrshort{IoT} in conjunction with industry 4.0 is also being hyped since a couple years. Gartner estimates an increase from 8.3 billion\footnote{short scale billion, which is $10^{9}$} to 20.4 billion devices. This leads to new requirements in respect to the security and way of communication of those devices - an aspect that may be solved through the use of blockchain technology.\cite{gartner.com_iot,BlockchainRevolution}

Specific smart contracts were developed in the scope of this bachelor's thesis, which can invoked through a so-called \acrfull{DAPP} from desktop computers as well as Android mobile devices. The functionality of the smart contracts were further integrated into an IoT demonstrator to exemplify a physical reaction to the result of the interaction with a smart contract.