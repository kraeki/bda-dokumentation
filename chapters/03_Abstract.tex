\chapter{Abstract d}
\label{cha:abstract_d}
Blockchain. Kryptowährungen. Dezentralisierte Datenbank. Internet of Things. Was sich für Laien anhört wie ein Auszug aus dem Computerduden, lässt die Herzen von so manchem Informatikstudenten höher schlagen. Bei diesen Begriffen handelt es sich nicht nur um Ausdrücke zur Geltendmachung der mentalen Überlegenheit durch Kenntnis von Fachbegriffen. Betont werden vor allem die rasant wachsende Anzahl vernetzter Geräte, die zunehmende Verteilung von Computersystemen \footnote{Referenz zu verteilten Energiesystemen, selbst fahrenden Autos o.Ä.?}, die immer grösser werdende Frage der Sicherheit derselben\footnote{Stichwort Globalisierung, Vernetzung} und die zunehmende Unlust von Unternehmen einander in der heutigen globalisierten Welt zu vertrauen\footnote{Kryptowährung \& BC}.

Pioniert wurde die Blockchain Technologie durch eine unter dem Pseudonym Satoshi Nakamoto bekannte Person. Diese implementierte im Jahr 2008 die heute immer noch bekannteste Blockchain mit gleich heissender Kryptowährung: Bitcoin. Neuere Blockchain Implementationen unterstützen neben Kryptowährungen auch sogenannte \emph{Smart Contracts}. Diese bestehen aus Programmcode, der in der Blockchain abgelegt wird. Dieser Programmcode kann beliebige Daten in diese verteilte Datenbank schreiben und wieder abfragen. Dadurch, dass dieser Code, und auch die Daten, verteilt und somit von jedermann einsehbar sind, können Verträge unmissverständlich und öffentlich verfügbar in einem von Maschinen lesbaren Format definiert werden. Änderungen können nicht im geheimen vorgenommen werden, da die Blockchain garantiert, dass jede dezentrale Datenbank denselben Stand aufweist. Dies fördert die Transparenz und beseitigt die Notwendigkeit für eine Drittpartei, der beide Vertragsparteien vertrauen\footnote{bspw. Bank, Notar o.Ä.}. Dadurch wird erreicht, dass beide Parteien zu ihrem besten Können auf die Erfüllung des Vertrags hin arbeiten.

Auch das Thema \acrshort{IoT} ist seit mehreren Jahren, nicht zuletzt dank dem Begriff Industrie 4.0, aller Munde. Bis 2020 schätzt Gartner einen Zuwachs von momentan 8.3 Milliarden auf 20.4 Milliarden Geräte. Dies birgt neue Anforderungen an die Sicherheit und Kommunikationsweise der Geräte mit sich - ein Aspekt, der eventuell durch die Blockchain-Technologie gelöst werden kann.\cite{gartner.com_iot,BlockchainRevolution}

Im Rahmen dieser Bachelorarbeit wurden spezifische Smart Contracts entworfen, die durch eine sogenannte \acrfull{DAPP} von Desktop Rechnern und Androidgeräten angesteuert werden können. Diese Smart Contracts wurden ebenfalls in einen IoT Demonstrator integriert, um eine physische Reaktion des Resultats eines Smart Contracts zu veranschaulichen.


\chapter{Abstract e}
\label{cha:abstract_e}

Abstract goes here
