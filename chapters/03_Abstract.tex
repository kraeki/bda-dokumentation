\chapter{Abstract d}
\label{cha:abstract_d}

Die \emph{Blockchain}-Technologie ist zurzeit stark im Trend. Sie benutzt verteilte und dezentrale Rechnernetzwerke und bietet dadurch bessere Transparenz und geringere Kosten im Vergleich zu traditionellen Methoden. Auch wird dadurch die Sicherheit erhöht, da die einzelnen Nodes sich gegenseitig versichern, wahrheitsgemäss zu handeln.\cite{BlockchainRevolution}

Pioniert wurde die Blockchain Technologie durch die bekannte bitcoin Blockchain. Neue Blockchain Implementationen unterstützen neben Crypto-Währungen auch sogenannte \emph{Smart Contracts}, die es einem Benutzer erlauben, Programmcode zu implementieren, der in der Blockchain abgelegt wird. Dadurch, dass dieser Code zusammen mit den Daten einsehbar ist, können so Verträge unmissverständlich und öffentlich einsehrbar definiert werden. Programme, die diese Smart Contracts verwenden, nennt man auch \emph{\acrfull{DAPP}}.\cite{BlockchainRevolution}

Auch das Thema \acrshort{IoT} ist seit mehreren Jahren in aller Munde und wächst in den kommenden Jahren stark an. Gartner schätzt einen Zuwachs von 20.4 Milliarden Geräten bis 2020. Dies bringt neue Anforderungen an die Sicherheit mit - ein Aspekt, der eventuell durch die Blockchain-Technologie erfüllt werden kann.\cite{gartner.com_iot,BlockchainRevolution}



\chapter{Abstract d}
\label{cha:abstract_d}

Abstract goes here
