\section{Arbeitsmethodik}
\label{sec:Arbeitsmethodik}
\begin{itemize}
    \item \textbf{Wie sind wir vorgegangen}
    \item \textbf{Folgende Kapitel auf derselben Stufe erklären (Recherche, Konzeption, Implementation)}
    \item \textbf{Verwendete Tools}
    \item \textbf{Open Source erwähnen (zeigt Engagement :D)}
\end{itemize}
Da in diesem Projekt eine konkrete Aufgabenstellung selbst zu erarbeiten war, und keine konkreten, lieferbaren Objekte vorgesehen waren, wurde zu Beginn eine explorative ad-hoc Methodik verfolgt. Dabei wurde im Projektteam wöchentlich der bisherige Fortschritt analysiert und weitergehende Schritte besprochen. So wurde gewährleistet, dass in der Anfangsphase eine geeignete Aufgabenstellung gefunden werden kann.
\par
Sobald die Aufgabenstellung gefunden wurde, wurde in ein iterativ- inkrementelles Modell gewechselt. Die wöchentlichen Besprechungen wurden beibehalten, um den Fortschritt des Produkts zu verfolgen.

\subsection{Verwendete Tools}
\paragraph{Koordination}
Da kein Projektmanagement betrieben wurde, finden sich hier auch keine tools.

\paragraph{Entwicklung}
VIM, VS Code etc?