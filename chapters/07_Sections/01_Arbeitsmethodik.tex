\section{Arbeitsmethodik}
\label{sec:Arbeitsmethodik}
\begin{itemize}
    \item \textbf{Wie sind wir vorgegangen}
    \item \textbf{Folgende Kapitel auf derselben Stufe erklären (Recherche, Konzeption, Implementation)}
    \item \textbf{Verwendete Tools}
    \item \textbf{Open Source erwähnen (zeigt Engagement :D)}
\end{itemize}
Da in diesem Projekt eine konkrete Aufgabenstellung selbst zu erarbeiten war, und keine konkreten, lieferbaren Objekte vorgesehen waren, wurde zu Beginn eine explorative ad-hoc Methodik verfolgt. Dabei wurde im Projektteam wöchentlich der bisherige Fortschritt analysiert und weitergehende Schritte besprochen. So wurde gewährleistet, dass in der Anfangsphase eine geeignete Aufgabenstellung gefunden werden kann (vgl. \ref{pm_cha:Projektmanagement}).

Sobald die Aufgabenstellung definiert wurde, wurde in ein iteratives Modell gewechselt. Die wöchentlichen Besprechungen im Projektteam wurden beibehalten, um den Fortschritt des Produkts zu verfolgen. Es wurde ein Anforderungskatalog geführt, der nach jeder wöchentlichen Besprechung überarbeitet wurde. Die Zeitplanung wurde anhand der definierten Meilensteine gemacht.

Die Anforderungen wurden vom Projektteam selbst erstellt und beinhalten grundlegende Funktionalität, die der Demonstrator umsetzen muss. Die Priorisierung der Anforderung hat keine Bedeutung, da der Demonstrator nur bei Umsetzung aller Anforderungen einsatzfähig ist. Daher rechtfertigt der Nutzen von detaillierten User Stories und Repriorisierung dieser, sowie Durchführung von Sprintplanung, Retrospektive und \dots den zusätzlichen Aufwand nicht.

\subsection{Verwendete Tools}
\paragraph{Koordination}
todo...

\paragraph{Entwicklung}
VIM, VS Code etc?


\subsection{Open Source}
Da die Ethereum Implementation \emph{geth}, die verwendete mobile Library für geth, \emph{status-go}, und auch ein Grossteil weiterer verwendeter Libraries\footnote{ethjsonrpc, web3js (teil von eth)} Open Source sind, wurde bei diesem Projekt viel auf die Zusammenarbeit mit Open Source Projekten gelegt. Beide Hauptprojekte wurden von github.com geforked, um Fehler einfacher nachvollziehen zu können. Dabei wurden auch Fehler gefunden, die an die ursprünglichen Entiwckler mitgeteilt wurden.


\subsubsection{https://github.com/ethereum/go-ethereum}
geth ist supi

\paragraph{Abonnieren von Whispers}
https://github.com/ethereum/go-ethereum/issues/14450
https://github.com/ethereum/go-ethereum/pull/14470

\subsubsection{https://github.com/status-im/status-go}
status-go wurde geforkt, da die Config für die mainnet und testchains statisch eingebunden wurde. ohne diese Änderung wäre es nicht möglich gewesen eine node auf einem mobilen gerät auf eine private Blockchain zu connecten.

\paragraph{Propagation von Transactions}
https://github.com/status-im/status-go/issues/169

\paragraph{Senden von Whispers}
https://github.com/status-im/status-go/issues/164
https://github.com/status-im/status-go/pull/163
