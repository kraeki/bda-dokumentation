\section{Arbeitsmethodik}
\label{sec:Arbeitsmethodik}
Da in diesem Projekt eine konkrete Aufgabenstellung selbst zu erarbeiten war, und keine konkreten, lieferbaren Objekte vorgesehen waren, wurde zu Beginn eine explorative ad-hoc Methodik verfolgt. Dabei wurde im Projektteam wöchentlich der bisherige Fortschritt analysiert und weitergehende Schritte besprochen. So wurde gewährleistet, dass in der Anfangsphase eine geeignete Aufgabenstellung gefunden werden kann (vgl. \ref{pm_cha:Projektmanagement}).

Sobald die Aufgabenstellung definiert wurde, wurde in ein iteratives Modell gewechselt. Die wöchentlichen Besprechungen im Projektteam wurden beibehalten, um den Fortschritt des Produkts zu verfolgen. Es wurde ein Anforderungskatalog geführt, der nach jeder wöchentlichen Besprechung überarbeitet wurde. Die Zeitplanung wurde anhand der vorab definierten Meilensteine gemacht.

Die Anforderungen wurden vom Projektteam selbst erstellt und beinhalten grundlegende Funktionalität, die im Demonstrator umgesetzt werden müssen. Die Priorisierung der Anforderung hat keine Bedeutung, da der Demonstrator nur bei Umsetzung aller Anforderungen einsatzfähig ist. Der zeitliche Aufwand für detaillierte User Stories, Repriorisierung dieser sowie Durchführung von Sprintplanung- und Retrospektiv-Sitzungen wird somit nicht durch zusätzlichen Nutzen gerechtfertigt.

\subsection{Verwendete Software}

\begin{table}[H]
\centering
\caption{Platformen und Sprachen}
\label{tbl:Platformen_Sprachen}
\begin{tabular}{@{}L{3cm}R{1cm}L{10cm}@{}}
\toprule
Name & Version & Komponente \\\midrule
Bash & 4.3 & Controller des Schliessmechanismus \\ \midrule
EcmaScript & 6 & Webapp \\ \midrule
html & 5 & Webapp \\ \midrule
Java & 1.6 & Android App \\ \midrule
\latex & - & Dokumenterstellung \\\midrule
Python & 2.7 & Doorman \\ \midrule
Solidity & 0.4.8 & Smart Contracts \\ \midrule
\end{tabular}
\end{table}

\begin{table}[H]
\centering
\caption{Entwicklungstools}
\label{tbl:Entwicklungstools}
\begin{tabular}{@{}L{3cm}R{1cm}L{10cm}@{}}
\toprule
Name & Version & Verwendungszweck \\\midrule
Android Studio & 2.3.2 & IDE für Android Apps\\ \midrule
git & 2.9.0 & Versionskontrolle \\ \midrule
github.com & - & Versionskontrolle und Zusammenarbeit (vgl. \ref{para:Versionskontrolle}) \\ \midrule
sharelatex.com & - & Erstellung und Kollaboration der formellem Dokumente\\ \midrule
VIM & 7.5 & Source Code Editor \\ \midrule
Visual Sudio Code & 1.12.2 & Source Code Editor \\ \bottomrule
\end{tabular}
\end{table}

\paragraph{Versionskontrolle}
\label{para:Versionskontrolle}
Aller entwickelte Programmcode ist unter der Organisation github.com/lokkit verfügbar. Das Projektteam benutzt folgende Pseudonyme auf github.com:

Hirzel Dominik: nnicksize

Schmid Andreas: kraeki


\subsection{Open Source}
\label{subsec:Open_Source}
Da die Ethereum Implementation \emph{geth}, die verwendete mobile Library für geth, \emph{status-go}, und auch ein Grossteil weiterer verwendeter Libraries\footnote{ethjsonrpc, Web3.js, VueJS} Open Source sind, wurde bei diesem Projekt viel auf die Zusammenarbeit mit Open Source Projekten gelegt. Beide Hauptprojekte wurden von github.com geforked, um Fehler einfacher nachvollziehen zu können. Dabei wurden auch Fehler gefunden, die an die ursprünglichen Entwickler mitgeteilt wurden.

\subsubsection{geth}
\emph{https://github.com/ethereum/go-ethereum}

Die Referenzimplementation geth wurde in diesem Projekt ursprünglich in Version 1.5.9 verwendet und basierend darauf wurden erste Gehversuche mit einer private Chain und Smart Contracts gemacht. Die Version wurde laufend dem aktuellen Entwicklungsstand angepasst und wurde zuletzt in der Version 1.6.2 verwendet.

\paragraph{Abonnieren von Whispers}
In der Konzipierung stellte sich heraus, dass Whisper v5 unterstützt werden musste (vgl. \ref{subsec:Real_Time_Kommunikationsschnittstelle}). Daher wurde die zwischenzeitlich veröffentlichte Version 1.6.2 von geth verwendet, da diese als erste Whisper v5 unterstützt. Vergangene Versionen von geth unterstützten nur die alte Version Whisper v2. Dabei wurde eine Inkonsistenz im Design der Subprotokolle von Ethereum und deren Implementation in geth festgestellt, worauf ein öffentliches Issue erstellt\footnote{https://github.com/ethereum/go-ethereum/issues/14450} und ein Pullrequest zur Behebung dieses Problemes gestellt\footnote{https://github.com/ethereum/go-ethereum/pull/14470} wurden.

\subsubsection{status-go}
\emph{https://github.com/status-im/status-go}

status-go wurde geforkt, da die Konfiguration für das Mainnet (Homestead) und Testnet (Ropsten) statisch darin definiert wurden. Diese Konfigurationen wurden angepasst, damit auf die private Lokkit Blockchain verbunden werden konnte. Ohne diese Änderung wäre es nicht möglich gewesen, eine Ethereum Node auf einem mobilen Gerät auf die private Lokkit Blockchain zu verbinden.

\paragraph{Propagation von Transactions}
Bei der Verwendung von status-go wurde erkannt, dass die gestartete Node auf einem Androidgerät erstellte Transaktionen nicht an andere Nodes weitergeleitet wird. Da dies ein essenzieller Bestandteil der Funktionalität von Lokkit ist, wurde ein öffentliches Issue dafür erfasst, das vom Entiwcklungsteam aufgenommen wurde\footnote{https://github.com/status-im/status-go/issues/169}.

\paragraph{Senden von Whispers}
In diesem Fall waren die Senden- und Empfangsfunktionen für Whisper v5 Nachrichten in der Version 1.6.2 von geth nicht kompatibel mit denselbigen aus status-go, wofür ein öffentliches Issue erfasst wurde\footnote{https://github.com/status-im/status-go/issues/164}. Als der Fehler vom Projektteam eruiert wurde, konnte ein pull request gestellt werden, der den Fehler behebt\footnote{https://github.com/status-im/status-go/pull/163}.
