\documentclass[bachelor,german]{hgbthesis}
% Zulässige Class Options: 
%   Typ der Arbeit: diplom, master (default), bachelor, praktikum 
%   Hauptsprache: german (default), english
%%------------------------------------------------------------

\graphicspath{{images/}}    % wo liegen die Bilder? 
\bibliography{literatur}  	% Angabe der BibTeX-Datei, % utf8-change

%%%  Acronym and Glossary
\usepackage[utf8]{inputenc}
\usepackage[acronym, toc]{glossaries}
\usepackage{listings}
%% listings configuration
\lstdefinelanguage{JavaScript}{
  keywords={typeof, new, true, false, catch, function, return, null, catch, switch, var, if, in, while, do, else, case, break},
  keywordstyle=\color{blue}\bfseries,
  ndkeywords={class, export, boolean, throw, implements, import, this},
  ndkeywordstyle=\color{darkgray}\bfseries,
  identifierstyle=\color{black},
  sensitive=false,
  comment=[l]{//},
  morecomment=[s]{/*}{*/},
  commentstyle=\color{purple}\ttfamily,
  stringstyle=\color{red}\ttfamily,
  morestring=[b]',
  morestring=[b]"
}
%% end listings configuration

\makenoidxglossaries

\newglossaryentry{maths}
{
    name=mathematics,
    description={Mathematics is what mathematicians do}
}

\newacronym{lcm}{LCM}{Least Common Multiple}
\newacronym{IoT}{IoT}{Internet of Things}
\newacronym{DAPP}{DAPP}{Decentralized Application}
\newacronym{EVM}{EVM}{Ethereum Virtual Machine}

%%%----------------------------------------------------------
\begin{document}
%%%----------------------------------------------------------

% Einträge für ALLE Arbeiten:
\title{lokkit - Einsatz von Blockchain und IoT}
\author{D.\ Hirzel\ und A.\ Schmid}
\studiengang{Informatik}
\studienort{Rotkreuz}
\abgabedatum{2017}{06}{09}	% {YYYY}{MM}{DD}
%\strictlicense  % erzeugt restriktive Lizenzformel

%%%----------------------------------------------------------
\frontmatter
\maketitle
\tableofcontents
%%%----------------------------------------------------------		

% \include{chapters/02_Selbststaendigkeitserklaerung} wird durch \maketitle bereits gemacht.
\chapter{Abstract d}
\label{cha:abstract_d}

Blockchain. Kryptowährungen. Dezentralisierte Datenbank. Internet of Things. Was sich für Laien anhört wie ein Auszug aus dem Computerduden, lässt die Herzen von so manchem Informatikstudenten höher schlagen. Bei diesen Begriffen handelt es sich nicht nur um Ausdrücke zur Geltendmachung der mentalen Überlegenheit durch Kenntnis von Fachbegriffen. Betont werden vor allem die rasant wachsende Anzahl vernetzter Geräte, die zunehmende Verteilung von Computersystemen \footnote{Referenz zu verteilten Energiesystemen, selbst fahrenden Autos o.Ä.?}, die immer grösser werdende Frage der Sicherheit derselben\footnote{Stichwort Globalisierung, Vernetzung} und die zunehmende Unlust von Unternehmen einander in der heutigen globalisierten Welt zu vertrauen\footnote{Kryptowährung \& BC}.

Pioniert wurde die Blockchain Technologie durch eine unter dem Pseudonym bekannte Person Satoshi Nakamoto. Dieser legte den Grundstein für diese sogenannten verteilten Ledger und implementierte im Jahr 2008 die heute immer noch bekannteste Blockchain mit gleich heissender Kryptowährung: Bitcoin. Neue Blockchain Implementationen unterstützen neben Kryptowährungen auch sogenannte \emph{Smart Contracts}. Diese erlauben es Benutzern Programmcode zu implementieren, der in der Blockchain abgelegt wird. Dieser Code kann beliebige Daten in diese verteilte Datenbank schreiben und wieder abfragen\footnote{Natürlich muss das bezahlt werden}. Dadurch, dass dieser Code, und auch die Daten, verteilt und somit von jedermann einsehbar sind, können Verträge unmissverständlich und öffentlich verfügbar in einem von Maschinen lesbaren Format definiert werden. Dies fördert die Transparenz und bedingt, dass beide Parteien zu ihrem besten Können auf die Erfüllung des Vertrags hin arbeiten.

Anwendungen, die diese Smart Contracts verwenden, nennt man \emph{\acrfull{DAPP}}. Diesebeschränken sich heute meist auf einfache Spiele\footnote{https://www.kingoftheether.com/thrones/kingoftheether/index.html}.


\cite{BlockchainRevolution}

Auch das Thema \acrshort{IoT} ist seit mehreren Jahren in aller Munde und wächst in den kommenden Jahren stark an. Gartner schätzt einen Zuwachs von 20.4 Milliarden Geräten bis 2020. Dies bringt neue Anforderungen an die Sicherheit mit - ein Aspekt, der eventuell durch die Blockchain-Technologie erfüllt werden kann.\cite{gartner.com_iot,BlockchainRevolution}

Im Rahmen dieser Bachelorarbeit wurden basierend auf ausgiebiger Recherche erwähnte Smart Contracts entwickelt und durch 


\chapter{Abstract e}
\label{cha:abstract_e}

Abstract goes here

%\printglossary[title=Abkürzungen und Definitionen, toctitle=List of terms]

\printnoidxglossary[type=\acronymtype]
 
\printnoidxglossary

%%%----------------------------------------------------------
\mainmatter         % Hauptteil (ab hier arab. Seitenzahlen)
%%%----------------------------------------------------------

\chapter{Aufgabenstellung}
\label{cha:Aufgabenstellung}

Im diesem Kapitel wird die Aufgabenstellung aufgezeigt, die im Rahmen dieser 
Arbeit ebefalls erstellt wurde. Einzig die Schlagworte \emph{Blockchain},
\emph{\acrfull{IoT}} und den Bau eines Demonstrators waren vorgegeben.

Die \emph{Blockchain}-Technologie ist zurzeit stark im Trend. Sie benutzt verteilte
und dezentrale Rechennetzwerke und bietet dadurch bessere Sicherheit und geringere 
Kosten im Vergleich zu traditionellen Methoden.

Neue Blockchains unterstützen neben Crypto-Währungen auch sogenannte \emph{Smart Contracts},
die es dem Benutzer erlauben, Programme für die Blockchain zu implementieren. Diese
Art von Programm nennt man auch \emph{\acrfull{DAPP}}.

Auch das Thema \acr{IoT} ist in aller Munde und wächst in den kommenden Jahren stark an. Gartner 
schätzt Zuwachs von 20.4 Milliarden neuen Geräten bis 2020\footnote{http://www.gartner.com/newsroom/id/3598917}. Dies bringt neue Anforderungen an
die Sicherheit mit - ein Aspekt, der eventuell durch die Blockchain-Technologie erfüllt
werden kann.

Im Rahmen dieser Bachelorarbeit soll ein System entwickelt werden, das den IoT und Blockchain Aspekt
umsetzt und demonstriert.

\begin{itemize}
    \item \textbf{ Wieso sind Blockchain und IoT wichtig? Gartner Hype Cycler}
    \item \textbf{ Wieso schlaegt die Schule dieses Thema vor? }
    \item \textbf{ Was soll mit dem Demonstrator erreicht werden? }
\end{itemize}

\section{Anforderungen an den Demonstrator}
\label{sec:Anforderungen an den Demonstrator}
\begin{itemize}
    \item Der Demonstrator soll auf dem Schulareal ausgestellt werden können. 
    \item Der Demonstrator soll keine Abhängigkeiten nach Aussen haben (kein Internet benötigen)
    \item Der Demonstrator soll wiederverwendbar sein.
    \item Der Aufbau des Demonstrators sowie die Inbetriebnahme muss dokumentiert sein.
\end{itemize}

\section{Konkrete Aufgabenstellung}
\label{sec:Konkrete Aufgabenstellung}
Es soll ein System entwickelt werden, um Schließfächer an Personen zu vermieten. Jedes Schließfach muss elektronisch ver- und entriegelt werden können.

Der Benutzer soll über eine Oberfläche ein freies Schließfach reservieren können. Sobald der Benutzer der aktuelle Mieter eines Faches ist, kann er das Schloss per Knopfdruck öffnen und schliessen.

Im Fokus steht die Umsetzung mit einer geeigneten Blockchain.

\section{Ziele}
\label{sec:Ziele}
\begin{itemize}
    \item \textbf{ Was sind die Ziele die durch diese Arbeit erreicht werden sollen? }
    \item \textbf{ Was ist das Ziel des Demonstrators? }
\end{itemize}

\begin{itemize}
    \item Demonstrieren der Vorteile der Blockchain
    \item \dots
\end{itemize}


\section{Erwartete Resultate}
\label{sec:Erwartete_Resultate}

Ein lauffähiger Prototyp des Schliessfachvermietsystems.
\chapter{Lösungsentwicklung}
\label{cha:Loesungsentwicklung}

\section{Recherche}
\label{sec:Blockchain}
\begin{itemize}
    \item \textbf{Informationssuche zu Beginn}
    \item \textbf{Was mit Blockchain alles gemacht werden kann}
    \item \textbf{Was wir mit Blockchain machen möchten}
    \item \textbf{technische Entscheidungen beschreiben}
    \item \textbf{Machbarkeitsanalyse lokkit}
    \item \textbf{Konzept verweist hierher für generelle infos und limitationen und Gründe für Entscheidungen}
\end{itemize}

referenz, verweise

\subsection{Verwendete Blockchain Implementation}
Dieses Projekt verwendet die open-source Blockchain-Implementation Ethereum\footnote{https://www.ethereum.org/} als Plattform für die Datenhaltung, die Business Logik in Form von Smart Contracts (vgl. \ref{subsec:Smart_Contracts} und \ref{subsec:lokkit_Smart_Contracts}) und einzige Interaktionsmöglichkeit für Benutzer mit dem System.

Ethereum ist die einzige Implementation einer Blockchain, die zum Start dieses Projektes eine funktionsfähige Plattform für eine Kryptowährung und Smart Contracts zur Verfügung stellt. Weitere Technologien, die in der Evaluationsphase analysiert wurden, sind Hyperledger, Bitcoin, Tendermint (nur ein consensus algo?), Nxt.

\subsubsection{Ethereum}
Die Finanzierung der Entwicklung der Ethereum Plattform durch ein Crowdfunding Projekt im Sommer 2014 ermöglicht. Der erste Release war ca. ein Jahr später am 30. Juli 2015. Die Implementation des lokkit Demonstrators wurde zu Beginn basierend auf der Version 1.5.9 von geth erstellt. Um neue Features wie Whisper v5 oder Mobile-Integration zu ermöglichen wurde die verwendete Version laufend an den aktuellen Entwicklungsstand der Ethereum Platform angepasst.
\paragraph{Konsensus}
Alle bisherigen Versionen des Ethereum Protokolls (Olympic, Frontier, Homestead) verwenden den \emph{Proof of Work} Algorithmus Ethash\footnote{https://github.com/ethereum/wiki/wiki/Ethash}, um Konsensus zu erreichen. Dies bedeutet, dass für jede Transaktion eine grosse Berechnung (proof) durchgeführt werden muss, bevor die Transaktion offiziell in die Blockchain eingetragen werden kann.

Für zukünftige Implementationen des Ethereum Protokolls war ein \emph{Proof of Stake} Algorithmus vorgesehen, um Konsensus zu erreichen, wurde jedoch kürzlich\footnote{} durch einen \emph{Proof of Authority} Algorithmus ersetzt. Grund dafür ist, dass die Kosten, eine Blockchain anzugreifen (bspw. um gewisse Blöcke neu zu schreiben durch erneute Berechnung der Hashes und zugehörigen Nonces), die einen Proof of Work Konsensus betreibt zusammen mit den Betriebskosten wachsen \footnote{https://www.coinmanual.com/proof\-stake/}\footnote{https://en.bitcoin.it/wiki/Proof\_of\_Stake}. Folglich kann ein Hochsicherheitssystem, wie es für Finanzdienste wie Banken oder Treuhänder benötigt würde, nur durch sehr hohe Betriebskosten realisiert werden. Ein wiederkehrendes Argument ist auch die Energie-Ineffizienz dieser Algorithmen. Aufgrund der benötigten hohen Rechenleistung sind Miner dazu geneigt ihre Rechenleistung zu konsolidieren und zu zentralisieren. Dies wird bei einem Anteil von 51\% Rechenleistung einer Entität zum Problem für den Konsensus. (\#TODO: see mail attachment "bda")

\paragraph{Solidity}
Solidity ist eine Programmiersprache, die auf der \acrfull{EVM} läuft, um Smart Contracts zu verfassen. Mehrere Sprachen konkurrieren mit Solidity, namentlich LLL und Serpent. Hierbei ist zu beachten, dass Solidity auf anderen Platformen wie Hyperledger Burrow durch eine Implementation der Ethereum VM ebenfalls lauffähig ist\footnote{https://github.com/hyperledger/burrow}.

\paragraph{Whisper}
\label{para:Whisper}
Whisper, auch analog zu der öffentlichen API \emph{shh} genannt, ist ein Kommunikationsprotokoll für \acrfull{DAPPs}. Nachrichten, die über das Whisper Protokoll verschickt werden, benutzen zur Übermittlung ebenfalls Ethereum Nodes, auf welchen das \emph{shh} Protokoll aktiviert wurde. Diese Nachrichten lösen keine Transaktionen auf der Blockchain aus. Wie auch die Transaktionen werden die Whisper Messages an alle Teilnehmer gesendet (broadcast), können aber mit einem \emph{topic} versehen werden, das es vereinfacht, erhaltene Nachrichten zu filtern. Jede Nachricht kann mit einer \acrfull{ttl} versehen werden, die angibt, wie lange die Nachricht verfügbar sein soll. Wenn diese Zeitspanne abläuft wird die Nachricht von den Ethereum Nodes nicht mehr weitergeleitet (Stale Message) und steht auf den Nodes, die die Nachricht bereits erhalten haben, nicht mehr zur Verfügung. So bietet das Whisper Protokoll eine einfache, leichtgewichtige Möglichkeit zur Echtzeit-Kommunikation für \acrshort{DAPPs}.

\#TODO: gehört das zu Konzeption? oder ganz streichen? Um das Whisper Protokoll auf einer Ethereum node zu aktivieren, muss geth mit dem \emph{-shh} Argument gestartet werden. Dies ermöglicht der jedoch Node erst, erhaltene Whisper Nachrichten von anderen Nodes weiterzuleiten. Damit die Whisper API über web3 zugänglich wird, ist zusätzlich der Eintrag \emph{shh}, in der mittels \emph{--rpcapi} angegebenen Liste der aktiven APIs, notwendig.
\begin{lstlisting}[language=bash,caption=Beispiel für die Aktivierung des shh Protokolls auf der Ethereum Node]
geth --shh
\end{lstlisting}
\begin{lstlisting}[language=bash,caption={Beispiel für die Aktivierung der web3, eth und shh API}]
geth --rpcapi "web3,eth,shh"
\end{lstlisting}
\begin{lstlisting}[language=bash,caption={Beispiel für die Aktivierung des shh Protokolls und der web3, eth und shh API}]
geth --rpcapi "web3,eth,shh"
\end{lstlisting}
\subparagraph{Whisper Identity}
\label{supara:Whisper_Identity}
\#TODO:obsolete. Um Whisper Nachrichten zu senden oder zu empfangen, kann eine Whisper Identity erstellt werden. Diese Identity wird während der Laufzeit des Programs gespeichert und beim beenden von bspw. \emph{geth attach} wieder gelöscht. Wird bei einer Nachricht ein Empfänger, in Form einer Whisper Identity, angegeben, wird die Nachricht verschlüsselt, sodass nur dieser die Nachricht entschlüsseln kann\footnote{http://web3js.readthedocs.io/en/1.0/web3-shh.html\#post}. Der Absender einer Nachricht, wie auch das Erstellen einer Whisper Identity an sich, ist optional. Das heisst, es können anonyme Nachrichten über Whisper verschickt werden, die keinerlei Rückschluss auf deren Ursprung machen lassen.

\subsubsection{Hyperledger}
Hyperledger ist ein Unterfangen mehrerer internationaler Firmen (\#TODOLinux Foundation, IBM, Intel etc), das zum Ziel hat, wichtige Funktionalität im Bereich von Blockchains zu erforschen und zu definieren, um schlussendlich einen industrieweiten offenen Standard für verteilte Ledgertechnologien zu erschaffen\footnote{https://wiki.hyperledger.org/community/welcomesheet}.

Zum Zeitpunkt des Starts dieses Projekts waren die Projekte 
\paragraph{Fabric}
\paragraph{Chaintool}
\paragraph{Burrow}

\subsection{Ethereum Testchain}
Die Ethereum Implementatoin stellt hart-kodiert zwei öffentliche Chain zur Verfügung. Zum einen die produktiv verwendete Chain \emph{Homestead}\footnote{}, auf der mit physischer Währung Ether gehandelt wird und die auch von mehreren Firmen produktiv verwendet wird\footnote{}. Zum anderen die Chain \emph{Ropsten}, die als aktuelle Testchain dient. Obwohl das Mining ebenfalls mit erheblicher Rechenleistung verbunden ist, kann Ether auf der Testchain nicht gegen \emph{echte} Währung eingetauscht werden. Grund daf¨r

\paragraph{Faucet Request}
\label{para:Faucet_Request}
Auf allen Testchains können sogenannte \emph{Faucet Requests}\footnote{https://blog.b9lab.com/when-we-first-built-our-faucet-we-deployed-it-on-the-morden-testnet-70bfbf4e317e}\footnote{https://ethereum.stackexchange.com/questions/84/what-public-test-networks-and-faucets-exist} gemacht werden, wodurch Ether erhalten werden kann. Dieser Ether wird von Entwicklern, die aktiv Mining betreiben, zur Verfügung gestellt (vgl. donate) und kann dann von weiteren Entwicklern verlangt werden. Meist ist ein Limit auf der Menge Ether pro Minute oder Adresse verhängt, das missbräuchliche Nutzung dieses Dienstes verhindern soll. Ziel dieser Faucets ist es, Entwicklern eine Möglichkeit zu geben, die öffentliche Testchain zu verwendet, ohne dabei selbst Mining betreiben zu müssen.

\subsection{Private Chain mit Ethereum}
Es ist möglich, unter Verwendung von Mobile Apps von Drittanbietern\footnote{Ethereum Status}, die \emph{Ropsten} Testchain anzusprechen. Durch Wunsch des Auftraggebers wurde eine Anbindung an die öffentliche \emph{Ropsten} Testchain ausgeschlossen, da die ständige Verfügbarkeit des Internets oder die Verfügbarkeit von Ether nicht gewährleistet sein kann. Folglich wurde eine private Blockchain auf der zur Verfügung gestellten Infrastruktur\footnote{Raspis} installiert. Um eine private Blockchain aufzusetzen wird ein Genesis Block benötigt und es müssen bestehende Nodes bekannt sein, die als peers hinzugefügt werden können.

\paragraph{Genesis Block}
In einer Blockchain wird der erste Block Genesis Block genannt. Hier werden initiale Kontostände (oft für die Entwickler oder Contributer) angelegt und unterschiedliche Konfigurationen bezüglich.
Um eine Verbindung zu einem Peer aufzubauen muss der Genesis Block auf beiden Nodes übereinstimmen. Wenn dies der Fall ist, kann der Synchronisierungsprozess beginnen.

\subsection{Konzept der Smart Contracts}
\label{subsec:Smart_Contracts}
''A computerized transaction protocol that executes the terms of a contract.``\cite{BlockchainRevolution}

Mit diesem Zitat kann grob das Ziel von Smart Contracts beschrieben werden. Verträge, die in einem maschinenlesbaren Format verfasst sind, können mit unmissverständlicher Präzision und ohne Interpretationsspielraum definiert werden. So ist es möglich, einen deterministischen, digitalen Vertrag zu verfassen.
Entgegengesetzt ist es nahezu unmöglich eine Menge von Smart Contracts angesichts einer unüblichen Situation deterministisch abzuarbeiten\footnote{http://www.ibtimes.co.uk/pwc-blockchain-expert-pinpoints-sources-ambiguity-smart-contracts-1575778}.

\subsubsection{Erstellung}
Um einen Smart Contract zu erstellen, wird dieser mittels einer geeigneten Programmiersprache formell definiert. Dieser geschriebene Programmcode kann durch eine Transaktion in die Blockchain eingesetzt werden, wobei Initialwerte für den Smart Contract angegeben werden. Man sprich hierbei vom erstellen einer Instanz des Smart Contracts, analog der Erstellung einer Instanz einer Klasse in der objektorientierten Programmierung. Bei der ausgelösten Transaktion gilt es zu beachten, dass dieser kein Empfänger angegeben wird; der Empfänger dieser Nachricht ist die Blockchain selbst. Wie bei jeder Transaktion eine gewisse Menge gas mitgegeben werden, damit die Operation abgeschlossen werden kann\footnote{echt? dachte das geht ohne explizit gas anzugeben}. Wenn der Code in die Blockchain eingesetzt wurde, erhält diese Instanz des Contracts eine Adresse, über die später mit dieser spezifischen Instanz interagiert werden kann.

Beim Erstellen eines Smart Contracts wird auch ein abi\footnote{Application Binary Interface} generiert. Dieses beschreibt die möglichen verfügbaren Attribute des Contracts und die Interaktionsmöglichkeiten (\#VGL. Funktionen) inklusive deren allfällige Parameter und Rückgabewerte (\#VGL. Interfacedeklaration in OO Sprachen).

\subsubsection{Lesezugriff}
Um ein Attribut auszulesen oder eine Funktion mit konstantem Rückgabewert auszuführen, kann diese über ein verfügbares Interface, wie die JavaScript console von geth, direkt aufgerufen werden. Bei Funktionen mit konstantem Rückgabewert ist zu beachten, dass diese nur auf der lokalen Node emuliert werden\footnote{} und es somit nicht möglich ist, eine Änderung in der Blockchain zu bewirken. Diese Funktionen haben folglich nur Lesezugriff. Allfällige Events, die in einer konstanten Funktion definiert wurden werden nicht ausgelöst.

\subsubsection{Schreibzugriff}
Wenn eine Funktion auf der Instanz aufgerufen wird, die eine Änderung in der Blockchain bewirkt, muss eine Transaktion ausgelöst werden. Der Sender der Transaktion muss dabei für die Kosten für gas und allfällige weitere Kosten (\#VGL. Bezahlbare Funktionen) aufkommen. Der Sender kann nicht nur ein Account sein, sondern auch ein weiterer Smart Contract, an dessen Adresse genügend Ether liegt, um die Transaktionskosten zu begleichen.

\paragraph{Bezahlbare Funktionen}
Einige Funktionen von Smart Contracts benötigen Ether, den man dem Aufruf hinzugibt. Dabei ist zu beachten, dass die gas Kosten und explizit gesendeter Ether nicht dasselbe sind. Wenn beispielsweise eine Dienstleistung oder Sache über einen Smart Contract verkauft werden soll, muss eine Menge Ether als Zahlungsmittel überweisen werden. Die Menge Ether ist im Smart Contract festgelegt und kann durch Inspektion des Programmcodes eingesehen werden. Sollte der Kunde zu wenig Ether schicken, kann der Smart Contract definieren, dass die Transaktion nicht erfolgreich ist und der Kunde sein Geld zurückerhält. Dazu kann die Transaktion selbst als nichtig erklärt werden und es muss nicht auf das Withdrawal Muster zuückgegriffen werden (vgl. Withdrawal Muster)

\subsubsection{Solidity}
Solidity ist eine Programmiersprache zur Erstellung von Smart Contracts auf der \acrfull{EVM}, die syntaktisch stark an JavaScript angelehnt ist. Sie kann auch auf anderen Blockchain Platoformen (wie Tendermint oder Counterparty für Bitcoin) verwendet werden. -> https://www.cryptocoinsnews.com/counterparty-brings-ethereum-smart-contracts-to-the-bitcoin-blockchain/\\Solidity wird, in Anlehnung auf den Ausdruck object-oriented, als contract-oriented bezeichnet, da die erstellenden Konstrukte in der Sprache \emph{contract} und nicht \emph{object} genannt werden. Inhaltlich beziehen sich die beiden Ausdrücke auf dasselbe Konzept.


\section{Konzeption}
\label{sec:Konzeption}
\begin{itemize}
    \item \textbf{Proof of concept zur Machbarkeitsanalyse aus vorherigem Kapitel}
    \item \textbf{Architektur, Aufbau inkl. Diagramme etc.}
    \item \textbf{Zu entiwckelnde Komponenten auflisten und Zuständigkeiten definieren. Bspw. Doorman ist eine Referenzimplementation, die es ermöglicht auf whispers zu hören. Alternativ kann anhand von folgender Beschreibung ein eigener doorman implementiert werden...}
    \item \textbf{Schnittstellen beschreiben. bspw. Whisper für webapp<->doorman Kommunikation erwähnen, aber \emph{nicht} deb eigenen Signaturmechanismus oder den Aufbau der ausgetauschten Nachrichten}
\end{itemize}

Die entwickelte DAPP ermöglicht es einem Owner ein Objekt über die Blockchain zu vermieten. Er kann dabei das Depot und die Mietkosten festlegen. Ein Renter kann dann dieses Objekt für eine bestimmte Zeit mieten. Die Abwicklung des Vertrages und dessen Kosten geschieht im Hintergrund in der Blockchain. Es gibt keinen Zwischenmann (z.B. Bank), der für die Überweisung zuständig ist, sondern das Geld fliesst direkt vom Renter zum Owner. 
Ein Objekt kann grundsätzlich alles sein: Ein elektrisches Fahrrad, eine Ferienwohnung oder einfach ein Schließfach. Es muss jedoch möglich sein ein Kontrollmechanismus anzubringen, der gleichzeitig ein Node in der Blockchain darstellt. Bei einem elektrischen Fahrrad wäre es z.B. möglich einen Minicomputer im Fahrgestellt zu montieren, welcher den Akku aktiviert oder deaktiviert und über eine Sim-Karte mit der Blockchain synchronisiert. Möchte ein Benutzer das Fahrrad verwenden, so müsste er dieses zuerst über die Blockchain mieten und erst dann kann er den Akku aktivieren. Die Mietkosten würden direkt dem Anbieter des Fahrrades überwiesen werden.

\begin{figure}
\centering
\includegraphics[width=.95\textwidth]{Mobile_Konzept}
\caption{Das Mobile-Konzept, dessen Umsetzung möglich ist}
\label{fig:Mobile_Konzept}
\end{figure}

\vspace{1em}
Als Proof of Concept wurde in dieser Arbeit einen Demonstrator entwickelt. Es handelt sich um ein Schliessfach-Vermietsystem, welches als Backend eine private Blockchain nutzt. Jedes Fach hat ein elektrisches Türschloss, welches über ein Raspberry PI durch die Blockchain kontrolliert wird. Ein Schließfach und ein Raspberry PI bilden zusammen eine Einheit (ein Node der privaten Blockchain). Das erste Schliessfach erstellt ausserdem ein WLAN Hotspot, auf den alle anderen Einheiten verbinden. Der Demonstrator verfügt über 3 Einheiten, welche sich zum Hotspot verbinden. Die Anzahl der Einheiten ist nicht beschränkt, solange sie in der Reichweite des WLAN Hotspots befinden. Alle Einheiten zusammen bilden die private Blockchain. 

\subsection{Interaktion mit dem Demonstrator}
\label{sec:Interaktion mit dem Demonstrator}

Ein Benutzer muss sich zuerst als einen weiteren Node mit der Blockchain verbinden. Anschliessend kann er über die entwickelte Webapp (DAPP) mit dem System interagieren. 

\vspace{1em}\noindent
Folgende Interaktionen sind möglich.

\vspace{1em}\noindent
\textbf{Als Owner}
\begin{itemize}
    \item Neues Rentable (Objekt) erfassen
    \item Bestehende Rentable deaktivieren/aktivieren/zerstören
\end{itemize}

\vspace{1em}\noindent
\textbf{Als Renter}
\begin{itemize}
    \item Nach Rentables suchen (Discover)
    \item Informationen und Verfügbarkeit eines Rentables prüfen
    \item Ein Rentable reservieren (rent)
\end{itemize}

\vspace{1em}\noindent
\textbf{Als aktueller Renter (current renter)}
\begin{itemize}
    \item Sperren und Entsperren des Rentables (Lock/Unlock)
    \item Zurückgeben des Schließfachs (Unclaim)
    \item Verlängern der aktuellen Reservation (Emergency)
\end{itemize}

\subsection{Regeln}
\begin{enumerate}
    \item Hat zum aktuellen Zeitpunkt niemand das Rentable reserviert, so ist der Owner gleich dem Renter.
    \item Es gibt keine Reservationsüberlappungen
    \item Eine aktive Reservation kann mittels \emph{Emergency} verlängert werden. Steht die Verlängerung im Konflikt mit einer nächsten Reservation, so wird der Startzeitpunkt der Reservation nach hinten verschoben. Wird im Fallbeispiel ... erläutert.
    \item ...
\end{enumerate}


\subsection{Fallbeispiele}
\label{sec:Fallbeispiele}
Die folgenden Fallbeispiele erläutern das Zusammenspiel der Interaktionen und die Verschiebung der priviligierten Rechte entlang der Zeitachse.

\subsubsection{Normal-Fall}
In Abb.~\ref{fig:Cases}\,(a) reserviert Benutzer A das Schliessfach von $t_{R1}$ bis $t_{R2}$ zum Zeitpunkt $t_0$. Das Deposit und die Mietkosten werden ihm somit abgezogen.
Zum Zeitpunt $t_{R1}$ wird Benutzer A zum \emph{Current Renter} und besitzt dadurch privilegierten Zugriff. Er kann jetzt das Schliessfach öffnen (Unlock) und schliessen (Lock). Nach einer gewissen Zeit noch vor Ablauf der Reservation, entschliesst sich Benutzer A das Schliessfach wieder abzugeben (Unclaim), da er es nicht mehr benötigt. Die Reservation wird somit auf diesen Zeitpunkt gekürzt und Benutzer A erhählt sein Deposit und die Mietkosten für die ungenutzte Zeit zurück. Die priviligierten Rechte werden umgehend wieder an den Owner abgegeben.

\begin{figure}
\centering\small
\setlength{\tabcolsep}{0mm}	% alle Spaltenränder auf 0mm
\begin{tabular}{c@{\hspace{12mm}}c} % mittlerer Abstand = 12mm
  \includegraphics[width=.45\textwidth]{Case-Normal} &
  \includegraphics[width=.45\textwidth]{Case-No-Unclaim} \\
  (a) & (b)
  \\[1em]	%vertical extra spacing (4 points)
  \multicolumn{2}{c}{\includegraphics[width=.45\textwidth]{Case-Emergency}}\\
  \multicolumn{2}{c}{(c)} 
\end{tabular}
%
\caption{Verschiedene Fälle -- 
Normal-Fall (a), Keine-Rückgabe (b),
Emergency-Fall (c).}
\label{fig:Cases}
\end{figure}

%\begin{figure}
%\centering
%\includegraphics[width=.95\textwidth]{Case-Normal}
%\caption{Normal-Fall}
%\label{fig:Case-Normal}
%\end{figure}

\subsubsection{Reservationsende ohne Betätigung der Rückgabe}
Wie im Normal-Fall beschrieben, ist Benutzer A momentaner Renter (current renter) des Schliessfaches. Anstelle jedoch das Schließfach wieder zurückzugeben, lässt er die Reservationszeit auslaufen. Zum Zeitpunkt $t_{R2}$ werden ihm die Rechte entzogen und er erhält keine Rückerstattung des Deposits. Es liegt also in Benutzer A seinem Interesse, alle eingeschlossenen Gegestände vor Ablauf der Reservation aus dem Schließfach zu nehmen und dieses zurückzugeben (Unclaim). Siehe Abb.~\ref{fig:Cases}\,(b) als Illustration.

\subsubsection{Emergency zum Verlängern der Reservationsdauer}
Der Benutzer A sollte immer einen Buffer in die Reservationszeit einplanen, damit er das Schließfach auch noch bei Zugsausfall oder einem Autounfall rechtzeitig vor Reservationsablauf erreichen kann. In äußersten Notfällen kann es trotzdem vorkommen, dass der Benutzer A nicht rechtzeitig seine Affären aus dem Schließfach nehmen kann und aus diesem Grund hat er die Möglichkeit eine \emph{Emergency} auszulösen. Die Emergency ist entsprechend kostenintensiv und sollte deswegen nur in Notfällen und abhängig vom Wert der Gegenstände im Schließfach in Erwägung gezogen werden. 
Die Emergency verhindert den Ablauf der aktuellen Reservation und überschreibt allenfalls nachfolgende Reservationen. Betroffene Reservationen erhalten einen Teil der Einnahmen durch die Emergency als Entschädigung. Die Abbildung \ref{fig:Cases}\,(c) veranschaulicht dieses Scenario.

\subsubsection{Fall von Schäden}
Benutzer A reserviert wie im Normal-fall ein Schliessfach. Zum Zeitpunkt t1 will er das Schliessfach öffen (Unlock), dieses reagiert jedoch nicht. Er muss nun Kontakt mit dem Owner aufnehmen und die Sache klären. Auf jeden Fall sollte er nun das Schliessfach zurückgeben (Unclaim) um die Kosten niedrig zu halten, im Falle, dass sich der Owner nicht auffinden lässt oder dieser kein Interesse zeigt.

Im Rahmen dieser Arbeit wurde auch ein Konzept überlegt, wie solche Probleme vermindert werden könnten. Durch Einführen eines Review Systems, wo Benutzer die Owner beurteilen können. Owner mit einer guten Bewertung (Score) wären dann vertrauenswürdiger.

\begin{itemize}
    \item \textbf{ Wie funktioniert unsre Loesung? }
    \item \textbf{ Was ermoeglicht unsere Loesung? }
    \item \textbf{ Wie sicher ist das System? }
    \item \textbf{ Was sind moegliche Angriffe? }
    \item \textbf{ Wie stehts mit der Verfuegbarkeit?}
    \item \textbf{ Gibt es Restrictions?}
    \item \textbf{ Was sind die Staerken und Schwaechen dieser Loesung? }
    \item \textbf{ Wie ist der Ablauf? Was geschieht wann?\ref{sec:Fallbeispiele}}
    \item \textbf{ Was kann der Benutzer machen? \ref{sec:Interaktion mit dem Demonstrator}}
    \item \textbf{ Wie sind die Regeln? Wer kann reservieren und wann? Was passiert, wenn die Reservation ablauft? Wer haftet? Was ist, wenn das Schliessfach beschaedigt ist und ich es nicht antreten kann?\ref{sec:Fallbeispiele}}
    \item \textbf{ Wie sieht das Kostenmodel aus? Beispiele und Scenarien?}
    \item \textbf{ Wer nutzt unsere Loesung? }
    \item \textbf{ Warum nutzt man unsere Loesung?}
    \item \textbf{ Was sind Scenarien, wo unsere Loesung zum einsatz kommen wuerde?}
    \item \textbf{ Wie ist der grobe Aufbau?}
    \item \textbf{ Was fuer Technologien setzen wir ein und weshalb Blockchain?}
    \item \textbf{ Wie ist die Interaktion mit dem Benutzer? Kontextdiagramm? Es gibt mehrere Benutzer und Rollen.}
    \item \textbf{ Was brauchts um diese Loesung um zu setzten? Smart Contracts, Blockchain, Hardware, WebUI }
    \item \textbf{ In welche Komponenten ist das System aufgeteilt. Was sind die eizelnen Funktionen?}
    \item \textbf{ Was sind die Abhaengikeiten unseres Systems?}
    \item \textbf{ Wie sehen die Schnittstellen aus?} 
    \item \textbf{ Was fuer Technologien werden eingesetzt?} 
    \item \textbf{ Wie sieht das Design der Komponenten aus? Erweiterbar? Wartbar?} 
    \item \textbf{ Wie sieht das User Interface aus?} 
    \item \textbf{ Welche Libraries verwenden wir? Was haben wir entwickelt?}
    \item \textbf{ Wie wird das System deployed?}
    \item \textbf{ Kann man updates fahren?}
    \item \textbf{ Schrittanleitung zur Installation?}
\end{itemize}
    

\section{Implementation (Komponenten)}
\label{sec:Implementation}
Nachfolgend werden getroffene Entscheidungen zum Entwurf und zur Implementation der Komponenten beschrieben. Angetroffene Probleme werden vorgestellt und und deren Lösung erklärt\footnote{Konkrete technische Hindernisse mit Verweise auf Anhang zu Open Source changes/contributions oder auch technische Limitation wie bspw. go-ethereum mobile hat keine whisper API}.

Allfällige Abweichungen zur Konzeption (vgl. \ref{sec:Konzeption}, bspw. teilweise oder fehlende Implementation, werden erwähnt und, wo angebracht, werden weitere Überlegungen und Möglichkeiten zur Erweiterung der Komponenten erläutert.

\subsection{Smart Contracts}
\label{subsec:Smart_Contracts}
Die Smart Contracts, die in Solidity implementiert wurden, werden nachfolgend erläutert. Diese Implementationen bilden das Rückgrat der Applikation. Sie bilden die Business Logik in Code Form ab und dienen als einzige Interaktionsmöglichkeit mir der Datenbank in der Blockchain.

(vgl. http://solidity.readthedocs.io/en/latest/introduction-to-smart-contracts.html)

\#WICHTIG!!
Erwähnen der shortcomings, da unsere Contracts eine nicht vordefinierte Anzahl Iterationen haben. Dies kann aufgrund des Block-Gas limits zu stalling führen. Auch lese-Operationen, die aus anderen Smart-Contracts aufgerufen werden können diesen zum stallen bringen. -> http://solidity.readthedocs.io/en/develop/security-considerations.html\#gas-limit-and-loops

\subsubsection{Generell}
\paragraph{Datenhaltung}
In einem Smart Contract können Daten in der Blockchain auf unterschiedliche Arten gespeichert werden: \emph{Storage}\footnote{http://solidity.readthedocs.io/en/latest/miscellaneous.html\#layout-of-state-variables-in-storage} oder \emph{Events} (auch \emph{Logs} genannt). Storage ist ein einfacher Schlüssel-Werte-Paar Speicher, worauf alle Instanzvariablen eines Smart Contracts gespeichert werden, die zwischen verschiedenen Funktionsaufrufen persistiert werden müssen (vgl. reservations Feld für Rentables). Ein Smart Contract kann nicht ausserhalb seines zugewiesenen Speicherbereichs operieren. Zugriffe auf diesen Speicher sind teuer. So soll verhindert werden, dass grosse Speichermengen in der Blockchain verwendet werden. \footnote{http://solidity.readthedocs.io/en/latest/introduction-to-smart-contracts.html?highlight=memory\#storage-memory-and-the-stack}. Ausgelöste Events werden in der Blockchain in Form von Transactionlogs gespeichert. Diese Logs sind sehr günstig (vgl. gas price) anzulegen, können aber von einem Smart Contract nicht gelesen werden \footnote{http://jonathanpatrick.me/blog/ethereum-compressed-text}. Über die web3 API kann auf die so erstellten Logs zugegriffen werden \#TODO: insert Beispiel.

Somit kann gesagt werden, dass wenn eine Instanz eines Smart Contracts persistent Daten speichern und diese Daten in einem späteren Funktionsaufruf abfragen möchte, zwingend Storage Speicher verwendet werden muss. Damit die Konsistenz der Reservationen eines mietbaren Objektes gewährleistet werden kann, muss der Rentable Smart Contract die Möglichkeit besitzen, bestehende Reservationen abzufragen. Daher ist es unumgänglich für den Rentable Smart Contracts Storage Speicher für die Datenhaltung der Reservationen zu verwenden. Auch für den Rentable Discovery Smart Contract ist es zwingen notwendig, dass die registrierten Rentable Instanzen abgefragt werden können. Somit ist auch bei diesem Storage Speicher für die Datenhaltung der Adressen zu verwenden.

\subsubsection{Rentable}
\label{subsubsec:Rentable}
Vgl. \ref{sys_subsubsec:Rentable}
\\Damit alle mietbaren Objekte logisch unabhängig von einander sind, wurde der primäre Smart Contract so entworfen, dass für jedes mietbare Objekt eine separate Instanz dessen erstellt werden muss (vgl. contract Instanz erstellen). Die in der Blockchain liegende Funktionalität beinhaltet nur das Mieten, und die damit verbundene Monetären Transaktionen. Der Smart Contract weist dabei keine lokkit-spezifische Funktionalität auf. Bei dem Design wurde darauf geachtet, die Funktionalität zu abstrahieren, um es Autoren zu ermöglichen, weitere Dienste zu erstellen, die auf dem Konzept der mietbaren Objekte basiert. Diese darauf aufbauenden Dienste können auch ausserhalb des lokkit Systems bestehen und können gänzlich inkompatibel zu diesem sein.

\paragraph{Withdrawal Muster}
Bei Transaktionen, die einen Account zum Ziel haben, werden die gas-Kosten automatisch vom Sender beigefügt (\#TODO: quelle+beispiel+formel). Ist das Ziel aber eine Funktion eines Smart Contracts (wie es bei Rentable.rent der Fall ist) oder ein Smart Contract selbst\footnote{Default Function: https://ethereum.stackexchange.com/questions/15420/how-to-send-ether-to-contracts-address/15450\#15450}, so kann der Smart Contract, der die Transaktion erhält, Code ausführen, der vom initialen Sender der Transaktion bezahlt werden muss (vgl. gas). Um dies zu verhindern, sollte nicht Ether an eine Adresse überwiesen werden, von der nicht bekannt ist, ob sie ein Smart Contract oder ein Account ist\footnote{https://github.com/ethereum/go-ethereum/wiki/Contracts-and-Transactions}.

\subparagraph{Beispiel}
Sei A eine Instanz eines Rentable Contracts, der das Withdrawal Muster implementiert. Wenn ein Depot zurückerstattet wird, überweist A Ether nicht direkt an eine Zieladresse, B. Stattdessen merkt sich A die Menge Ether, die B zusteht und stellt die Möglichkeit zur Verfügung, dass B Ether vom Smart Contract abheben kann. Dies findet mittels einer Transaction auf \emph{withdraw} statt, die von B ausgelöst werden muss. Die gas-Kosten für die primär von B ausgeführte \emph{withdraw} Funktion sind vernachlässigbar und werden durch die zurückerhaltene Menge Ether kompensiert. Der Vorteil, wenn die Transaktion von B gestartet werden muss, ist, dass B für die etwaigen zusätzlichen gas-Kosten aufkommen muss, sollte B ebenfalls ein Smart Contract sein. 

\#möglicherweise ein bildli zeichnen oder beispielcode einfügen

\paragraph{Re-Entrancy}
Hierbei handelt es sich um ein Muster, das angewendet wird, wenn in einem Smart Contract ein interner Zustand existiert, der die Menge an Ether speichert, der \emph{anderen} zusteht und eine Möglichkeit besteht, diese Menge Ether zu entnehmen (\#VGL. Withdrawal Muster). Beispiele für zurückzuerstattenden Ether kann bspw. Depots wie bei lokkit oder ein Gebot in einer Auktion\footnote{http://solidity.readthedocs.io/en/develop/security-considerations.html\#re-entrancy} sein.

\subparagraph{Beispiel}
Sei A eine Instanz eines Rentable Contracts, der diesen erwähnten internen Zustand speichert und somit das Withdrawal Muster implementiert. Unter der Annahme, dass die Zieladresse für die Rückerstattung ein beliebiger Smart Contract B ist, kann B A angreifen. Wenn eine Transaktion ausgelöst wird, wird jeweils nicht nur der Betrag in Ether überwiesen, sondern, sollte das Ziel ein Smart Contract sein, auch Code ausgeführt. Das heisst, dass B auf eine eingehende Überweisungen reagieren kann\footnote{solidity default function: function () payable\{\}} und ihrerseits erneut die Transaktionsfunktion von A aktivieren kann (\#VGL. Withdrawal Muster), bevor die Kontrolle an den ursprünglichen Aufruf von A zurückgegeben wird\footnote{daher re-entrancy, da die Funktion mehrere Male ausgeführt wird}. Wenn der Zustand von A noch nicht geändert wurde, wird der Betrag erneut gesendet, bis A keinen Ether mehr zur Verfügung hat.

\subparagraph{Implementationsbeispiel}
Um mehrfaches Überweisen von Ether zu verhindern, ist es unvermeidbar, zuerst den internen Status des Vertrags zu ändern, bevor Ether überwiesen wird. Die \emph{total} zur Verfügung stehende Menge Ether eines Smart Contracts wird in der Blockchain gehalten und ist Teil des Konsensus. Es ist somit nicht möglich, dass mehr Ether ausgegeben wird als in dem jeweiligen digitalen Vertrag vorhanden ist. 

Im Codebeispiel \ref{lst:withdraw_erroneous} kann die \emph{send} Anweisung in Zeile zwei mehrmals ausgeführt werden, da das deposit erst zurückgesetzt wird, sobald die Übertragung erfolgreich war. Dabei ist zu bemerken, dass die Funktion .send(...) die Kontrolle an die default-Funktion\footnote{vgl. default-funktion} des Zielvertrags übergibt und deren Code ausgeführt wird.
\begin{lstlisting}[language=javascript,caption={fehlerhaftes Code Snippet},label={lst:withdraw_erroneous}]
    function withdraw() {
        if (msg.sender.send(deposit[msg.sender])) {
            deposit[msg.sender] = 0;
        }
    }
\end{lstlisting}

Im Codebeispiel \ref{lst:withdraw_good} wehrt die dritte Zeile die Re-Entrancy Attacke ab, da der interne Kontostand zurückgesetzt wird, bevor die Transaktion ausgelöst wird. Auch hier wird der default-Funktion\footnote{vgl. default-funktion} des Zielvertrags die Kontrolle übergeben. Im Fall, dass diese aber erneut die \emph{withdraw} Funktion aufruft, wird diese keine erneute Transaktion auslösen (msg.sender.transfer(0) wird keine Transaktion auslösen\footnote{}).
\begin{lstlisting}[language=javascript,caption={empfohlenes Code Snippet},label={lst:withdraw_good}]
    function withdraw() {
        var currentRefund = deposit[msg.sender];
        deposit[msg.sender] = 0;
        msg.sender.transfer(currentRefund);
    }
\end{lstlisting}

\paragraph{Weitergehende Überlegungen}
Mögliche Verbesserungen bzgl. der momentanen Implementation.

\subparagraph{Migrations}
Um neue oder geänderte Funktionalität in Smart Contracts zur Verfügung zu stellen, muss die Datenhaltung von der Businesslogik entkoppelt werden. Grund dafür ist, dass wenn ein neuer Smart Contract als Interaktionsmöglichkeit in der Bockchain steht, Daten, die durch Transaktionen in der Blockchain entstanden sind, nicht einfach übernommen werden können, wie beispielsweise Reservationen. Diese Daten könnten prinzipiell durch den Besitzer des Objekts migriert werden. Das würde aber bedeuten, dass in der Blockchain der Besitzer des Objekts im Namen seiner Mieter Reservationen tätigt, was seinerseits wiederum eine Sicherheitslücke und mögliches Missbrauchspotential seitens des Besitzers bedeutet.

Truffle Migrations?


\subsubsection{RentableDiscovery}
Für jede Interaktion mit einem mietbaren Objekt muss dessen Adresse bekannt sein. Das heisst, um administrative Aufgaben am Objekt wahrzunehmen, diese zu mieten oder mit ihnen über das Whisper Protokoll zu interagieren. Diese benötigte Adresse kann manuell gefunden und eingegeben werden, beispielsweise durch eine Aufschrift auf dem Objekt oder durch einen QR Code\footnote{vgl webapp}. Alternativ zu diesem Ablauf wurde eine Utility entworfen, die das Erstellen und Finden von mietbaren Objekten erleichtert.
\\Der RentableDiscovery Contract stellt eine Möglichkeit zur Verfügung, Rentable Contracts in der Blockchain einzutragen, ohne direkt mit dem Rentable Contract zu interagieren, ähnlich dem Factory Pattern aus der objektorientierten Programmierung\footnote{http://www.oodesign.com/factory-pattern.html}. Diese Factory hat den Vorteil, dass damit erstellte Rentables direkt der Discovery bekannt sind und gefunden werden können. Es existiert auch die Möglichkeit, bestehende Rentables nach deren manuellen Erstellung nachzutragen. Die zu bezahlenden Kosten in gas sind beim neuen Erstellen niedriger als beim nachträglichen Hinzufügen, da nicht zuerst die Existenz der Adresse geprüft werden muss.

\#TODO: einfügen einer Formel für gas-Kosten wäre nicht schlecht.

\paragraph{Weitergehende Überlegungen}
Durch Verwendung eines QR Code Scanners im Webapp wird diese Funktionalität obsolet gemacht. Sie wird in lokkit nicht mehr auf der Oberfläche eingesetzt. Sie kann optional verwendet werden, um eine Übersicht über bestehende Rentables zu haben, und dabei nur eine einzelne Adresse (die der RentableDiscovery) zu speichern.

In Zukunft könnten bei Bedarf Rentables mittels dieser RentableDiscovery in logische Gruppen unterteilen werden, um die Darstellung zu erleichtern. Weiter könnte ein dynamisches Angebot von Rentables zur Verfügung gestellt, indem Abfragen nicht über manuelle QR Codes sondern über eine (oder mehrere) RentableDiscovery gestellt werden.

\subsection{Real-Time Kommunikationsschnittstelle}
Für Implementationsdetails, vgl. \ref{subsec:Real_Time_Kommunikationsschnittstelle}

Da die Block Time in der Ethereum Blockchain etwa 12 Sekunden beträgt\footnote{https://blog.ethereum.org/2014/07/11/toward-a-12-second-block-time/}, sind Transaktionen nicht nützlich, um in Echtzeit mit einem gemieteten Objekt zu interagieren. Auch die dabei anfallenden Kosten einer Transaktion\footnote{Diese Kosten sind gering für einfaches Signaling mit Smart Contracts.} sind zu begleichen und speziell bei wiederholtem Senden der Signale nicht zu verachten. Auch dient die Persistenz in der Blockchain all dieser Interaktionen keinem Nutzen, da aus dem Smart Contract bekannt ist, dass der Benutzer Zugriff auf das Gerät hat.

Daher wurde, um mit mietbaren Objekten (vgl. \ref{subsec:lokkit_Smart_Contracts}) während des gemieteten Zeitraums in Echtzeit zu interagieren, das Protokoll Whisper v5\footnote{vgl. \ref{para:Whisper}}\footnote{Auch Whisper v2 wurde angeschaut, aufgrund der nicht vorhandenen Unterstützung in der go-ethereum mobile API fallen gelassen und für die status-go mobile API mit Whisper v5 Unterstützung ausgetauscht. VGL. Anhang} verwendet. Whisper v5 erlaubt die Übermittlung von textbasierten Nachrichten mit eingebautem Sicherheitsmechanismus zur Verschlüsselung und optionale Signatur von Nachrichten. Diese Nachrichten müssen entweder symmetrisch oder asymmetrisch verschlüsselt werden\footnote{https://github.com/ethereum/go-ethereum/wiki/Whisper-Usage}.

Da Whisper v5 die einzige Möglichkeit zur Echtzeitkommunikation in der Ethereum Blockchain ist, gibt es keine Alternativen dazu.

\subsubsection{Verteilung}
\ref{subsubsec:Verteilung}
Das Whisper v5 Protokoll setzt für jede Nachricht einen Schlüssel (symmetrisch oder asymmetrisch) und ein Topic (4 bytes) voraus. Whisper Abos müssen mindestens ebenfalls einen Schlüssel und ein Topic als Filter beinhalten. Um neuen Teilnehmern zu er möglichen auf diese Nachrichten zu hören und solche zusenden, wird ein symmetrischer Schlüssel mit dem Passwort \emph{lokkit} erstellt. Dieser Schlüssel muss nicht geheim gehalten werden, da der Inhalt der Nachricht nicht sicherheitsrelevant ist\footnote{Sollte in Zukunft die Anforderung an verschlüsselte Commands gestellt werden, vgl Anhang \#TODO: generate asymmetric key + in smart contract schreiben. --> vgl. Sicherheitsaspekte. Müsste transaction auslösen von doorman}. Als Topic werden die ersten 4 Bytes der sha3 Checksumme der Adresse des Mietbaren Objektes mitgeschickt. Basierend auf diesen beiden Eigenschaften kann der Doorman (vgl. \ref{subsec:Doorman}) die für die angeschlossenen Geräte relevanten Nachrichten filtern.

\subsubsection{Sicherheitsaspekte}
\label{subsubsec:Sicherheitsaspekte}
Nur die Verwendung kryptographisch sicherer Funktionen und Funktionalität (vgl. whisper, sha3, ec crypto) garantiert kein sicheres System. Auch wenn Whisper v5 garantiert\footnote{}, dass die übermittelten Nachrichten nach gängigen Sicherheitsstandards verschlüsselt werden, können Design-Lücken bei der Implementation der Real-Time Kommunikationsschnittstelle es einem Angreifer ermöglichen, das System auf eine nicht erwünschenswerte Art zu missbrauchen.

\emph{In den folgenden Paragraphen wird Vorwissen um die Real-Time Kommunikationsschnittstelle vorausgesetzt. Vgl. \ref{sys_subsec:Real_Time_Kommunikationsschnittstelle} für technische Details der Schnittstelle und deren Implementation.}

\paragraph{Signatur des Message Objektes}
Der Sender/Empfänger Mechanismus von Whisper ist nicht an die Accounts/Adressen der Blockchain gekoppelt. Da mietbare Objekte (in Form von Smart Contracts) von einer Adresse gemietet werden, wird ein Sicherheitsmechanismus benötigt, um sicher zu stellen, dass der Envelope von einer Entität geschickt wurde, die im Besitz des in der Blockchain angegebenen Accounts ist. Dafür wurde die Funktion \emph{web3.eth.sign(...)}\footnote{} verwendet, um das Message Objekt zu signieren. Damit diese Funktion einen Wert zurück liefert, muss der Private Key des angegebenen Accounts auf der lokalen Node vorhanden sein und dieser Account muss entsperrt sein. Sind diese Vorbedingungen erfüllt, liefert \emph{web3.eth.sign(...)} für einen gegebenen Input (in diesem fall das Message Objekt) einen 65 Byte Hash, Digest genannt, zurück. Wenn bspw. ein Empfänger der Nachricht Funktion \emph{web3.personal.ecRecover(...)} dasselbe Message Objekt und den erhaltenen Digest als Parameter übergibt, gibt diese Funktion die öffentliche Adresse zurück, die den Digest erstellt hat. Diese öffentliche Adresse kann dann mit der eingetragenen Adresse im Smart Contract abgeglichen werden.

\paragraph{Replay Attack}
\label{para:Replay_Attack}
Folgend wird der einzige bekannte technische Angriff auf die Real-Time Schnittstelle beschrieben und erläutert wie dieser, unter Verwendung der in Ethereum eingebauten kryptographischen Funktionen\footnote{web3.eth.sign, web3.eth.recover und shh asymmetric keys}, verhindert werden kann. Jegliche Angriffsvektoren, die die Referenzimplementation \emph{geth}, die \acrshort{EVM} oder natürliche Personen (d.h. Social Engineering) betreffen, werden nicht beschrieben oder entschärft.

Jeder Whisper Envelope muss mit einem symmetrischen oder asymmetrischen Schlüssel verschlüsselt werden und benötigt ein Topic (vgl. \ref{subsubsec:Verteilung}).
Auch für jedes Abonnement (subscription) von Whispers muss ein symmetrischer oder asymmetrischer Schlüssel und mindestens ein (oder mehrere) Topic als Filter angegeben werden. Da der Symmetrische Schlüssel für lokkit Envelopes öffentlich ist, ist er nicht für kryptographische Sicherheit, sondern nur für die Filterung der Envelopes, zu verwenden. Aus diesem Grund ist es für jeden Teilnehmer, der den Envelope auf seiner Node übermittelt, möglich, diesen aufzuzeichnen, die Payload zu extrahieren und erneut zu schicken (vgl. https://en.wikipedia.org/wiki/Replay\_attack). Diese Art der Weiterleitung funktioniert, wenn die gesamte Payload inklusive Digest weitergeleitet würde und somit keine erneute Signatur des Message Objektes nötig wäre. Somit wäre es möglich, eine Payload zu erhalten, die nicht von der signierenden Entität sondern von einer Drittperson gesendet wurde. Um sicher zu gehen, dass die Nachricht von derselben Entität gesendet wurde, die auch den Digest des Message Objektes erstellt hatte, wird das \emph{key} Attribut (vgl. \ref{sys_para:Key}) ebenfalls in das Message Objekt eingebettet. Dies ist derselbe Schlüssel, der zur Signatur des Whisper Envelopes verwendet wurde. Diese beiden Schlüssel können verglichen werden (vgl. \ref{sys_subsubsec:Verteilung}, um sicher zu gehen, dass der Sender der Whisper Nachricht auch derselbe ist, wie der, der den Digest erstellt hat.

\#todo: bild. dringend. den obigen Abschnitt checke ich selbst nicht, 30 Minuten nachdem ich ihn gelesen hab'...

\subsection{Doorman}
\label{subsec:Doorman}
Doorman ist eine mögliche Implementation in Python 2.7, die es ermöglicht, ein IoT Gerät an die erwähnten Smart Contracts (vgl. \ref{subsec:lokkit_Smart_Contracts}) und das Protokoll Whisper v5 (vgl. \ref{para:Whisper}) inkl. der definierten Schnittstelle (vgl. \ref{subsec:Real_Time_Kommunikationsschnittstelle}) anzubinden. Doorman verbindet sich mit einer angegebenen Ethereum Node, lädt die Informationen zum angegebenen Smart Contract und wartet auf Mitteilungen (vgl. \ref{subsec:Konfiguration}). Wenn eine Nachricht erhalten und der Absender erfolgreich verifiziert wurde, wird der angegebene Befehl in der Shell ausgeführt. Als einziges Argument erhält der Befehl den Inhalt des command Feldes der erhaltenen Mitteilung enthält (vgl. \ref{subpara:Command}).\#TODO: address und command, da mehrere rentables am selben doorman hängen können?

\subsubsection{Konfiguration}
\label{subsec:Konfiguration}
Die Konfiguration des Doorman wird aus einer yml Datei\footnote{https://fdik.org/yml/} gelesen. Die Datei besitzt ein root mit namen \emph{doorman} und Attribute für die Ethereum Node, verwendete Rentable Smart Contract Adresse und auszuführendes Script bei Erhalt einer validen Nachricht.
\begin{lstlisting}[language=yml,caption={Beispielkonfiguration für Doorman}]
doorman:
    node_ip: 127.0.0.1
    node_rpc_port: 8545
    rentable_addresses:
        - "0xf16801293f34fc16470729f4ac91185595aa6e10"
        - "0xf16801293f34fc16470729f4ac91185595aa6e10"
    script: open_close_device.bat
\end{lstlisting}

\paragraph{node\_ip}
Die Url der Ethereum Node, zu der eine Verbindung aufgebaut werden soll. Diese Node muss das shh Protokoll und auch die shh API aktiviert haben (vgl. \ref{para:Whisper}). Die Adresse ist als IP (\#TODO: ist das möglich? oder als aufzulösender hostname) anzugeben. Informationen zu protokoll wie \emph{http://} sind auszulassen.
\paragraph{node\_rpc\_port}
Der Port, auf dem Doorman zu der Ethereum Node verbinden soll. 
\paragraph{rentable\_address}
Die Adressen der Smart Contracts vom typ Rentable (vgl. \ref{subsec:Rentable}), auf deren Nachrichten gehört werden sollen. Wird eine falsche Adresse angegeben, werden Befehle nicht wie erwartet ausgeführt.
\paragraph{script}
Das script, das ausgeführt werden soll im Format
\begin{lstlisting}[language=yml,caption={Beispielkonfiguration für Doorman}]

\end{lstlisting}

\subsubsection{Wieso Python? Wieso Python 2.7?}


\section{Arbeitsmethodik}
\label{sec:Arbeitsmethodik}
\begin{itemize}
    \item \textbf{Wie sind wir vorgegangen}
    \item \textbf{Folgende Kapitel auf derselben Stufe erklären (Recherche, Konzeption, Implementation)}
    \item \textbf{Verwendete Tools}
    \item \textbf{Open Source erwähnen (zeigt Engagement :D)}
\end{itemize}
Da in diesem Projekt eine konkrete Aufgabenstellung selbst zu erarbeiten war, und keine konkreten, lieferbaren Objekte vorgesehen waren, wurde zu Beginn eine explorative ad-hoc Methodik verfolgt. Dabei wurde im Projektteam wöchentlich der bisherige Fortschritt analysiert und weitergehende Schritte besprochen. So wurde gewährleistet, dass in der Anfangsphase eine geeignete Aufgabenstellung gefunden werden kann (vgl. \ref{pm_cha:Projektmanagement}).

Sobald die Aufgabenstellung definiert wurde, wurde in ein iteratives Modell gewechselt. Die wöchentlichen Besprechungen im Projektteam wurden beibehalten, um den Fortschritt des Produkts zu verfolgen. Es wurde ein Anforderungskatalog geführt, der nach jeder wöchentlichen Besprechung überarbeitet wurde. Die Zeitplanung wurde anhand der definierten Meilensteine gemacht.

Die Anforderungen wurden vom Projektteam selbst erstellt und beinhalten grundlegende Funktionalität, die der Demonstrator umsetzen muss. Die Priorisierung der Anforderung hat keine Bedeutung, da der Demonstrator nur bei Umsetzung aller Anforderungen einsatzfähig ist. Daher rechtfertigt der Nutzen von detaillierten User Stories und Repriorisierung dieser, sowie Durchführung von Sprintplanung, Retrospektive und \dots den zusätzlichen Aufwand nicht.

\subsection{Verwendete Tools}
\paragraph{Koordination}
todo...

\paragraph{Entwicklung}
VIM, VS Code etc?


\subsection{Open Source}
Da die Ethereum Implementation \emph{geth}, die verwendete mobile Library für geth, \emph{status-go}, und auch ein Grossteil weiterer verwendeter Libraries\footnote{ethjsonrpc, web3js (teil von eth)} Open Source sind, wurde bei diesem Projekt viel auf die Zusammenarbeit mit Open Source Projekten gelegt. Beide Hauptprojekte wurden von github.com geforked, um Fehler einfacher nachvollziehen zu können. Dabei wurden auch Fehler gefunden, die an die ursprünglichen Entiwckler mitgeteilt wurden.


\subsubsection{https://github.com/ethereum/go-ethereum}
geth ist supi

\paragraph{Abonnieren von Whispers}
https://github.com/ethereum/go-ethereum/issues/14450
https://github.com/ethereum/go-ethereum/pull/14470

\subsubsection{https://github.com/status-im/status-go}
status-go wurde geforkt, da die Config für die mainnet und testchains statisch eingebunden wurde. ohne diese Änderung wäre es nicht möglich gewesen eine node auf einem mobilen gerät auf eine private Blockchain zu connecten.

\paragraph{Propagation von Transactions}
https://github.com/status-im/status-go/issues/169

\paragraph{Senden von Whispers}
https://github.com/status-im/status-go/issues/164
https://github.com/status-im/status-go/pull/163


\chapter{Schlussfolgerung}
\label{cha:Schlussfolgerung}
\begin{itemize}
    \item \textbf{Wie beurteilen Sie die Blockchain-Technologie? Generell? im Bezug auf IoT? Gibt's offensichtliche Hindernisse?}
    \item \textbf{Was wuerden Sie im Projekt anders machen?}
    \item \textbf{Was hat am meisten Zeit gekostet?}
    
    \item \textbf{Sehen Sie Zukunft in dieser technologie?}
    \item \textbf{}
\end{itemize}

\section{Ausblick}
\label{sec:Ausblick}
\begin{itemize}
    \item \textbf{Ist es denkbar den Demonstrator produktive einzusetzen? Ist es ein reales Produkt, dass eine Firma einsetzen wollen wuerde?}
    \item \textbf{Was wird sich von der Technologie her noch aendern?}
    \item \textbf{Was sind die Challenges fuer die Zukunft?}
\end{itemize}
Blockchain für realtime Anwendungen wahrscheinlich nicht geeignet?
\par
Anonymität kann auf gefährlich sein, da man nicht weiss, von wem man eine Dienstleistung kauft. -> Wie im Darknet kann es sein, dass eine Dienstleistung bezahlt wird, aber nicht erhalten wird.
Gibt bei zentralisierten Angeboten (wie bspw. isoliert an einem Bahnhof oder in einer Hochschule) keinen ersichtlichen Mehrwert gegenüber einer ebenfalls zentralisierten Anwendung. Einige Nachteile wie bspw. Latenz, Verfügbarkeit oder Vertrauenswürdigkeit des Angebots sind sogar ersichtlich.

\subsection{Weitere Gedanken}
Bei Recherche gefundene Konzepte beinhalten auch bspw. Elektroautos an einem Lichtsignal zu tanken. Ist nicht wirklich machbar, da 12 Sekunden Blockcycle zu lange sind, um zu garantieren, dass eine Leistung bezogen werden kann (Transaktion ist erst sicher, sobald der block gemined ist...)



\section{Lessons Learned}
\label{sec:Lessons_Learned}

\subsection{Dominik Hirzel}
Die Abschlussarbeit war natürlich sehr spannend und sexy....

Bekanntschaft mit dem Nachtwächter natürlich eine Bereiecherung :)

\subsection{Andreas Schmid}
Ich bin de Andi. Zum glück hani mich für s Studium ade HSLU entschiede, susch hätti mir no müsse müe gä...

\include{chapters/09_Quellen_Tabellen_Abbildungsverzeichnisse}
%%%----------------------------------------------------------
%%%Anhang
\appendix
\include{appendix/anhang_a}	% Inhalt der CD-ROM/DVD

%%%----------------------------------------------------------
\MakeBibliography
%%%----------------------------------------------------------

%%%Messbox zur Druckkontrolle
\include{messbox}

\end{document}
